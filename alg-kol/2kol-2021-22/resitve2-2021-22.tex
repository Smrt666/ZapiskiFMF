\documentclass[a4,11pt]{article}

\usepackage{amsmath, amsthm, amssymb, amsfonts}
\usepackage[utf8]{inputenc}
\usepackage[T1]{fontenc}
\usepackage[slovene]{babel}
\usepackage{thmtools}
\usepackage{lmodern}
\usepackage[colorlinks=false]{hyperref}
\usepackage{textcomp}
\usepackage{enumitem}

\usepackage{floatrow}
\usepackage{wrapfig}

\usepackage{graphicx}
\usepackage{tikz}
\usepackage{tkz-euclide}
\usepackage{setspace}
\usepackage{geometry}
\usepackage{float}
\usepackage{framed}
\usepackage{mathrsfs}

\newcommand{\degree}{^\circ}
\newcommand{\R}{\mathbb{R}}
\DeclareMathOperator{\im}{im}
\DeclareMathOperator{\Lin}{Lin}

\title{Rešitve 2.~kolokvija iz algebre 2021/2022}
\author{}
\date{}

\begin{document}

\maketitle

\section*{1.~naloga}
    Na množici \(V = \left(0, \infty\right) \times \left(0, \infty\right)\) definiramo predpisa
    \[\left(x, y\right) \oplus \left(u, v\right) = \left(xu, yv\right), \ \lambda\cdot \left(x, y\right) = \left(x^\lambda, y^\lambda\right) \ \text{ za } x, y, u, v \in \left(0, \infty\right) \text{ in } \lambda \in \R.\]
    S tem postane množica \(\left(V, \oplus, \cdot\right)\) vektorski prostor nad \(\R\). Tega ni potrebno dokazovati.
    \begin{enumerate}[label=(\alph*)]
        \item Dokaži, da je \(\mathcal{B} = \{\left(e, 1\right), \left(1, e\right)\}\) baza vektorskega prostora \(V\).

        Definirajmo
        \[ \mathcal{A}\left(x, y\right) = \left(xy^2, \frac{x^2}{y} \right) \space \text{ za } \left(x, y\right) \in V.\]
        \item Dokaži, da je \(\mathcal{A}: V \to V\) linearna preslikava.
        \item Določi matriko linearne preslikave \(\mathcal{A}\) glede na bazo \(\mathcal{B}\)
    \end{enumerate}

\subsection*{Rešitev}
    \begin{enumerate}[label=(\alph*)]
        \item Naj bo \((x, y) \in V\) poljuben. Velja:
        \[(x, y) = \left(e^{\ln{x}}, e^{\ln{y}}\right) = \left(e^{\ln{x}}, 1\right) \oplus \left(1, e^{\ln{y}}\right) = \ln{x} \cdot (e, 1) \oplus \ln{y} \cdot (1, e)\]
        Pri tem sta \(x\) in \(y\) pozitivna, zato je logaritem definiran. Skalarja \(\ln x\) in \(\ln y\) sta iz polja \(\R\), torej je \(\mathcal{B}\) res baza prostora \(V\).
        
        \item Naj bosta \(\left(x, y\right), \left(u, v\right) \in V\) in \(\alpha, \beta \in \R\). 
        \begin{multline*}
            \mathcal{A}\left(\alpha \cdot \left(x, y\right) \oplus \beta \cdot \left(u, v\right)\right) =  
            \mathcal{A}\left(x^\alpha u^\beta, y^\alpha v^\beta\right) = \left(x^\alpha u^\beta \cdot \left(y^\alpha v^\beta\right)^2, \frac{\left(x^\alpha u^\beta\right)^2}{y^\alpha v^\beta}\right) \\
            = \left(x^\alpha y^{2\alpha}u^{\beta}v^{2\beta}, \frac{x^{2\alpha}u^{2\beta}}{y^\alpha v^\beta}\right)
            = \alpha \cdot \left(xy^2, \frac{x^2}{y}\right) \oplus \beta \cdot \left(uv^2, \frac{u^2}{v}\right)
            = \alpha \cdot \mathcal{A}\left(x, y\right) \oplus \beta \cdot \mathcal{A}\left(u, v\right).
        \end{multline*}

        \item Izračunajmo kam se slikajo bazni vektorji:
        \[\mathcal{A}\left(e, 1\right) = \left(e, e^2\right) = \left(e, 1\right) \oplus 2 \cdot \left(1, e\right)\]
        \[\mathcal{A}\left(1, e\right) = \left(e^2, \frac{1}{e}\right) = 2\cdot\left(e, 1\right) \oplus \left(-1\right)\cdot\left(1, e\right)\]
        Matrika linearne preslikave \(\mathcal{A}\) glede na bazo \(\mathcal{B}\) je torej
        \[A_{\mathcal{B},\mathcal{B}} = \begin{bmatrix}
            1 & 2 \\
            2 & -1
        \end{bmatrix}.\]
    \end{enumerate}

\section*{2.~naloga}
    Dana je preslikava \(\Phi: \R^{n \times n} \to \R^{n \times n}\) s predpisom
    \[\Phi\left(A\right) = A + \lambda A ^ \top,\]
    kjer je \(\lambda \in \R\) izbrani parameter. Določi jedro in sliko preslikave \(\Phi\) v odvisnosti od parametra \(\lambda\).

\subsection*{Rešitev}
    Da poiščemo jedro preslikave \(\Phi\), moramo rešiti enačbo
    \[A + \lambda A ^ \top = 0.\]
    To je ekvivalentno enačbi \(A = -\lambda A ^ \top\). Enačbo lahko transponiramo in dobimo 
    \(A ^ \top = -\lambda A\). Vstavimo v prejšnjo enačbo in dobimo  \(A = \lambda^2 A\). Če je \(A \ne 0\), potem vsebuje vsaj en neničeln element \(a_{ij}\).
    Za ta element mora veljati \(a_{ij} = \lambda^2 a_{ij}\), kar je možno le, če je \(\lambda = \pm 1\).
    Če \(\lambda \ne \pm 1\), potem je jedro preslikave \(\Phi\) enako \(\{0\}\), slika pa je enaka celotnemu
    prostoru \(\R^{n \times n}\) (to sledi iz dimenzijske enačbe).
    
    Če je \(\lambda = 1\), potem je jedro enako množici matrik, ki zadoščajo enačbi:
    \[A = - A^\top.\]
    Torej so jedro vse matrike \(A = \left[a_{ij}\right]\), kjer velja \(a_{ij} = -a_{ji}\) za vse \(i, j\).
    Po diagonali so torej ničle, elementi pod diagonalo pa nasprotno enaki elementom nad diagonalo.
    Slika preslikave \(\Phi\) je enaka množici simetričnih matrik. \(\im \Phi\) je podmnožica simetričnih matrik:
    \[\Phi\left(A\right) = A + A^\top = \left[a_{ij} + a_{ji}\right] = \left[a_{ji} + a_{ij}\right] = \left(\Phi\left(A\right)\right)^\top.\]
    Poleg tega je tudi vsaka simetrična matrika v sliki, saj za vsako simetrično matriko \(A\) lahko vzamemo
    matriko z njenimi elementi, ki so nad diagonalo, polovično vrednost elementov na diagonali, ostale pa postavimo na \(0\). Ta matrika
    se slika v \(A\), saj je \(\Phi\left(B\right) = \left[b_{ij} + b_{ji}\right] = A\). 
    Diagonalni elementi se seštejejo 
    \(b_{ii} + b_{ii} = \frac{1}{2}a_{ii} + \frac{1}{2}a_{ii} = a_{ii}\). Ostali pa se
    seštejejo \(b_{ij} + b_{ji} = a_{ij} = a_{ji}\), saj za \(i > j\) velja 
    \(b_{ij} = 0\) in \(b_{ji} = a_{ji}\).

    Za \(\lambda = -1\) je jedro enako množici simetričnih matrik, slika pa 
    množici matrik \(\left[a_{ij}\right]\), da velja \(a_{ij} = - a_{ji}\) za vse \(i, j\).
    Torej ničle so na diagonali, elementi pod diagonalo pa so nasprotno enaki elementom nad diagonalo.
    Izračun slike in jedra za \(\lambda = -1\) je podoben kot za \(\lambda = 1\).

\section*{3.~naloga}
    Podana sta podprostora \(U,V \le \R^5\), kjer je 
    {\small\[
        U = \left\{\left(c_3 + c_4, -c_2, c_4, c_1 - c_2, c_0 + c_1 + c_3 + c_4\right) | \ 
        p\left(x\right) = c_4 x^4 + c_3 x^3 + c_2 x^2 + c_1 x + c_0, p\left(0\right) = p'\left(0\right) = p\left(1\right) = 0\right\}
    \]}
    in 
    \[
        V = \Lin\left\{\left(-1, 1, 0, 1, 1\right), \left(-2, 2, 1, 2, 2\right), \left(-1, 1, 1, 1, 1\right)\right\}.
    \]
    Določi dimenzije in baze prostorov \(U\), \(V\), \(U + V\) in \(U \cap V\).

\subsection*{Rešitev}
    Najprej iz pogojev ugotovimo naslednje: \begin{enumerate}
        \item \(p\left(0\right) = 0\) pomeni, da je konstantni člen \(c_0 = 0\).
        \item \(p'\left(0\right) = 0\) pomeni, da je koeficient pri \(x\) enak \(c_1 = 0\).
        \item \(p\left(1\right) = 0\) pomeni, da je vsota koeficientov enaka \(0\), torej \(c_0 + c_1 + c_2 + c_3 + c_4 = 0\).
        \item iz prejšnjih točk sledi, da je \(c_3 + c_4 = - c_2\).
    \end{enumerate}
    Torej je prostor \(U\) enak množici vektorjev oblike 
    \(\left(- c_2, - c_2, c_4, - c_2, -c_2\right)\), kjer je \(c_2, c_4 \in \R\).
    Potem je \(U = \Lin\left\{\left(-1, -1, 0, -1, -1\right), \left(0, 0, 1, 0, 0\right)\right\}\).
    Ta dva vektorja sta linearno neodvisna, torej tvorita bazo prostora \(U\). Dimenzija prostora \(U\) je \(2\).

    Da dobimo bazo prostora \(V\), moramo ugotoviti ali so podani vektorji linearno neodvisni.
    \[
        \begin{bmatrix}
            -1 & 1 & 0 & 1 & 1 \\
            -2 & 2 & 1 & 2 & 2 \\
            -1 & 1 & 1 & 1 & 1
        \end{bmatrix} \sim \begin{bmatrix}
            1 & -1 & 0 & -1 & -1 \\
            0 & 0 & 1 & 0 & 0 \\
            0 & 0 & 1 & 0 & 0
        \end{bmatrix} \sim \begin{bmatrix}
            1 & -1 & 0 & -1 & -1 \\
            0 & 0 & 1 & 0 & 0 \\
            0 & 0 & 0 & 0 & 0
        \end{bmatrix}
    \]
    Z odštevanjem 1.~vrstice od 2.~in 3.~ugotovimo, da je tretji vektor linearno odvisen od prvega in drugega.
    Torej je baza prostora \(V\) enaka \(\left\{\left(-1, 1, 0, 1, 1\right), \left(-2, 2, 1, 2, 2\right)\right\}\),
    dimenzija prostora \(V\) pa je \(2\).

    Da dobimo bazo prostora \(U + V\) lahko v dobljeno matriko vstavimo vektorje iz baze prostora \(U\).
    \[
        \begin{bmatrix}
            1 & -1 & 0 & -1 & -1 \\
            0 & 0 & 1 & 0 & 0 \\
            -1 & -1 & 0 & -1 & -1 \\
            0 & 0 & 1 & 0 & 0
        \end{bmatrix} \sim \begin{bmatrix}
            1 & -1 & 0 & -1 & -1 \\
            0 & 0 & 1 & 0 & 0 \\
            0 & -2 & 0 & -2 & -2 \\
            \ 
        \end{bmatrix} \sim \begin{bmatrix}
            1 & 0 & 0 & 0 & 0 \\
            0 & 0 & 1 & 0 & 0 \\
            0 & 1 & 0 & 1 & 1 \\
            \
        \end{bmatrix}
    \]
    Če odštejemo 1.~vrstico od 3.~in 2.~vrstico od 4.~dobimo 3 linearno neodvisne vektorje, ki tvorijo bazo prostora \(U + V\).
    Baza prostora \(U + V\) je torej enaka \(\left\{\left(1, 0, 0, 0, 0\right), \left(0, 0, 1, 0, 0\right), \left(0, 1, 0, 1, 1\right)\right\}\).
    Dimenzija prostora \(U + V\) je \(3\).

    Delo za računanje baze prostora \(U \cap V\) lahko olajšamo z uporabo dimenzijske enačbe.
    \begin{align*}
        \dim\left(U\right) + \dim\left(V\right) & = \dim\left(U + V\right) + \dim\left(U \cap V\right) \\
        2 + 2 & = 3 + \dim\left(U \cap V\right) \\
        \dim\left(U \cap V\right) & = 1
    \end{align*}
    Torej je dimenzija preseka prostorov \(U\) in \(V\) enaka \(1\). Baza tega prostora ima torej en element,
    en tak element se je pojavil že v bazi za \(U\) in kasneje v matriki za računanje baze prostora \(V\),
    zato je baza prostora \(U \cap V\) enaka \(\left\{(0, 0, 1, 0, 0)\right\}\).

    Če tega ne opazimo zapišemo splošna vektorja v obeh podprostorih in ju izenačimo:
    \[
        \left(-a, a, b, a, a\right) = \left(-c, -c, d, -c, -c\right)
    \]
    Iz tega sledi, da je \(a = c\), \(a = -c\) in \(b = d\). \(a\) in \(c\) sta potem \(0\),
    za \(b = d\) pa lahko vzamemo poljubno vrednost. Torej je baza preseka prostorov \(U\) in \(V\) enaka
    \(\left\{(0, 0, 1, 0, 0)\right\}\).

\section*{4.~naloga}
    Dokaži, da je množica 
    \[
        M = \left\{\left(f_1, f_2, \ldots, f_n\right) | \ f_1, f_2, \ldots, f_n \in \left\{-1, 1\right\}\right\}
    \]
    ogrodje vektorskega prostora \(\R^n\).

\subsection*{Rešitev}
    Množica \(M\) ima \(2^n\) elementov in je podmnožica \(\R^n\), zato je dovolj poiskati
    eno bazo za \(\R^n\), ki je podmnožica množice \(M\). Označimo z \(F_i = \left(1, 1, \ldots, 1, -1, 1, \ldots, 1\right)\)
    vektor, ki ima na \(i\)-tem mestu vrednost \(-1\), drugje pa \(1\). Množica \(M\) vsebuje vse vektorje oblike \(F_i\) 
    za \(i \in \left\{1, 2, \ldots, n\right\}\). Tudi \(\left(-1, -1, \ldots, -1\right) \in M\).
    Zdaj lahko poiščemo bazo:
    \[
        \begin{bmatrix}
            -1 & -1 & -1 & \cdots & -1 \\
            -1 & 1 & 1 & \cdots & 1 \\
            1 & -1 & 1 & \cdots & 1 \\
            1 & 1 & -1 & \cdots & 1 \\
            \vdots & \vdots & \vdots & \ddots & \vdots \\
            1 & 1 & 1 & \cdots & -1
        \end{bmatrix} \sim \begin{bmatrix}
            -1 & -1 & -1 & \cdots & -1 \\
            -2 & 0 & 0 & \cdots &  \\
            0 & -2 & 0 & \cdots & 0 \\
            0 & 0 & -2 & \cdots & 0 \\
            \vdots & \vdots & \vdots & \ddots & \vdots \\
            0 & 0 & 0 & \cdots & -2
        \end{bmatrix} \sim \begin{bmatrix}
            -1 & -1 & -1 & \cdots & -1 \\
            1 & 0 & 0 & \cdots & 0 \\
            0 & 1 & 0 & \cdots & 0 \\
            0 & 0 & 1 & \cdots & 0 \\
            \vdots & \vdots & \vdots & \ddots & \vdots \\
            0 & 0 & 0 & \cdots & 1
        \end{bmatrix}
    \]
    Če prvo vrstico odštejemo od vseh ostalih, potem pa jih delimo z \(-2\), 
    dobimo vektor \(\left(-1, -1, \ldots, -1\right)\) in standardne bazne vektorje za \(\R^n\).
    Prva vrstica je linearna kombinacija ostalih, če jo odstranimo, dobimo dobro poznano bazo za \(\R^n\).
    Torej \(\Lin M\) vsebuje standardno bazo za \(\R^n\), torej je \(M\) ogrodje za \(\R^n\).
\end{document}
