\documentclass[a4paper,12pt]{article}
\usepackage[slovene]{babel}
\usepackage[utf8]{inputenc}
\usepackage[T1]{fontenc}
\usepackage{lmodern}

\usepackage{url}
\usepackage{graphicx}
\usepackage{xcolor}
\usepackage{amsmath}
\usepackage{amsthm}
\usepackage{amssymb}

\title{kolokvij neki neki}
\author{Matija Derganc}

\begin{document}
\section*{1.~naloga} Naj bosta $a$ in $b$ (ne nujno različni) realni števili. Izračunaj naslednjo tripasovno
determinanto velikosti $n \times n$:
$$
\begin{vmatrix}
    a+b & ab & & & \\
    1 & a+b & ab \\
      & \dots & \dots & \dots \\
     &  &  & 1 & a+b & ab \\
     & & & & 1 & a+b
\end{vmatrix}	$$
\subsection*{Rešitev}
Determinanto razvijemo po prvem stolpcu: 

\begin{align*}
(a+b)\begin{vmatrix}
    a+b & ab & & & \\
    1 & a+b & ab \\
      & \dots & \dots & \dots \\
     &  &  & 1 & a+b & ab \\
     & & & & 1 & a+b
\end{vmatrix} -1
\begin{vmatrix}
    ab & 0 & & & \\
    1 & a+b & ab \\
      & 1 & a+b & ab \\
      & \dots & \dots & \dots \\
     &  &  & 1 & a+b & ab \\
     & & & & 1 & a+b
\end{vmatrix}
\end{align*}
Vidimo, da je leva determinanta enaka začetni, le za eno vrstico in stolpec manjša. Poimenujemo začetno determinanto
$D_n$ in levo $D_{n-1}$, desno stran pa razvijemo po prvi vrstici.
\begin{align*} (a+b)D_{n-1}-ab \begin{vmatrix}
    a+b & ab & & & \\
    1 & a+b & ab \\
      & \dots & \dots & \dots \\
     &  &  & 1 & a+b & ab \\
     & & & & 1 & a+b
\end{vmatrix} 
\end{align*}   
Dobljeni izraz  na desni pa je enak $D_{n-1}$. Imamo torej rekurzivno enačbo $D_n = (a+b)D_{n-1} - ab D_{n-2}$  \\
Nadaljujemo s postopkom reševanja rekurzivnih formul, tako da vstavimo namesto $D_n$ parameter $\lambda^n$
in obe strani deljimo z $\lambda^{n-2}$
Dobimo enačbo:
\begin{align*}
    \lambda^2 = (a+b)\lambda - ab \\
    \lambda^2 - (a+b)\lambda + ab = 0
\end{align*} 
Uporabimo formulo za ničle kvadratne funkcije:
\begin{align*}
    \lambda = \frac{a+b \pm \sqrt{(a+b)^2-4ab}}{2} \\
    \lambda = \frac{a+b \pm \sqrt{a^2-2ab+b^2}}{2} \\
    \lambda = \frac{a+b \pm \sqrt{(a-b)^2}}{2} \\
    \lambda_1 = \frac{a+b + a-b}{2} = a \\
    \lambda_1 = \frac{a+b -a+b}{2} = b 
\end{align*}
$\sqrt{(a-b)^2}$ smo lahko nadomestili z $\pm (a-b)$, ker se $\pm$ pojavi že pred korenom in je rezultat enak, ne glede na to ali je $a-b$ pozitivno ali negativno število.
Ker imamo dve rešitvi, je torej iskana determinanta linearna kombinacija potenc rešitev: $D_n = x a^n + yb^n$ \\
Neznana skalarja $x$ in $y$ poiščemo spomočjo enačb: \\ $D_1 = a+b, D_2 = (a+b)^2 -ab = a^2 +ab +b^2$
\\ 
\begin{align*}
    a+b = D_1 = xa + yb \\
    x = \frac{a+b-yb}{a}
\end{align*}
Vstavimo x v drugo enacbo :
\begin{align*}
    a^2 + ab + b^2 =D_2 = xa^2+yb^2 = \frac{a+b-yb}{a}a^2 + yb^2 = \\
    a(a+b-yb)+yb^2 = a^2 + ab -yab +yb^2 = a^2 + ab + yb(b-a)
\end{align*}
Odštejemo $a^2, ab$:
\begin{align*}
    b^2 = yb(b-a)\\
    b = y(b-a) \\
    y = \frac{b}{b-a}
\end{align*}
Ker smo obe strani delili z $b$, obravnavamo še primer, kjer je $b=0$. V tem primeru je matrika diagonalna in je njena determinanta enaka $a^n$
Nato izrazimo še $x$:
\begin{align*}
    x = \frac{a+b-yb}{a} = \frac{a+b-\frac{b}{b-a}b}{a} = \frac{\frac{(a+b)(b-a)-b^2}{b-a}}{a} = 
    \frac{\frac{-a^2}{b-a}}{a} = \frac{a}{a-b}
\end{align*}
Podobno obravnavamo primer, ko je $a = 0$, kjer je determinanta $b^n$.

Rešitev je torej: 
\begin{center}
    \large{$ \frac{a}{a-b} a^n+ \frac{b}{b-a} b^n$ }
\end{center}
\pagebreak


4. Naj bosta $A$ in $B$ enako veliki kvadratni kompleksni matriki, za kateri velja $AB=0$. 
Pokaži, da je za vsako naravno število $k$ sled matrike $(A+B)^k$ enaka vsoti sledi matrik $A^k$ in $B^k$.\\
\textbf{Rešitev:}
Pomagamo si z lastnostmi sledi matrike: \\$tr(A+B) = tr(A) + tr(B)$ in $tr(AB) = tr(BA)$. Nato $(A+B)^k$ razvijemo:
\[(A+B)^k = (A+B)(A+B)(A+B)^{k-2} = (A^2 + AB + BA + B^2)(A+B)^{k-2}\] 
Uporabimo dejstvo, da je $AB = 0$.
\[(A^2 + AB + BA + B^2)(A+B)^{k-2} = (A^2 + BA + B^2)(A+B)^{k-2} \]
Vidimo, da bodo v nadaljevanju vsi členi, ki vsebujejo $AB$ enaki 0 (v naslednjem koraku $A^2B, BAB$). Edini členi, ki ne vsebujejo $AB$ v tem vrstnem redu,
bodo $A^k, BA^{k-1}, B^2A^{k-2}, \dots, B^{k-1}A, B^k $. To lahko dokažemo z indukcijo, a menim, da je dovolj jasno tudi brez.
Nato uporabimo lastnost sledi:
\begin{align*}tr(A^k+BA^{k-1}+B^2A^{k-2}+\dots+B^{k-1}A+B^k) =  \\
tr(A^k)+ tr(BA^{k-1}) +tr(B^2A^{k-2})+ \dots +tr(B^{k-1}A)+tr(B^k) 
\end{align*}
Vidimo, da vsi členi v vsoti, razen $tr(A^k)$, $tr(B^k)$ vsebujejo $BA$ Uporabimo še lastnost, da sled ohranja produkte matrik,
\[ tr(B^iA^j)= tr(B^{i}A^{j-1}A) = tr(AB^iA^{j-1} ) = tr(ABB^{i-1}A^{j-1})=tr(0 \cdot B^{i-1}A^{j-1})=tr(0)=0\] 
Torej bodo v vsoti vsi členi, razen $A^k$ in $B^k$ enaki 0 in bo vsota enaka $A^k + B^k$



\end{document}