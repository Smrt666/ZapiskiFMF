\documentclass[a4,11pt]{article}

\usepackage{amsmath, amsthm, amssymb, amsfonts}
\usepackage[utf8]{inputenc}
\usepackage[T1]{fontenc}
\usepackage[slovene]{babel}
\usepackage{thmtools}
\usepackage{lmodern}
\usepackage[colorlinks=false]{hyperref}
\usepackage{textcomp}
\usepackage{enumitem}

\usepackage{floatrow}
\usepackage{wrapfig}

\usepackage{graphicx}
\usepackage{tikz}
\usepackage{tkz-euclide}
\usepackage{setspace}
\usepackage{mathrsfs}

\newcommand{\degree}{^\circ}
\newcommand{\R}{\mathbb{R}}
\DeclareMathOperator{\im}{im}
\DeclareMathOperator{\Lin}{Lin}

\title{Rešitve 4.~kolokvija iz algebre 2012/2013}
\author{}
\date{}

\begin{document}

\maketitle

\section*{1.~naloga}
    Linearni preslikavi \(A: \mathbb{C}^5 \to \mathbb{C}^5\) v standardni bazi ustreza matrika
    \[
        \begin{bmatrix}
            a & 0 & 0 & 0 & -b \\
            0 & a & 0 & 0 & -b \\
            0 & 0 & a & 0 & -b \\
            0 & 0 & 0 & a & -b \\
            b & b & b & b & a \\
        \end{bmatrix}.
    \]
    \begin{enumerate}[label=(\alph*)]
        \item Izračunaj njene lastne vrednosti.
        \item Za katere pare \(a, b \in \mathbb{C}\) je \(A\) unitarna preslikava?
    \end{enumerate}

\subsection*{Rešitev}
    Lahko izračunamo karakteristični polinom matrike \(A\):
    \begin{align*}
        & \det(A - \lambda I) = \begin{vmatrix}
            a - \lambda & 0 & 0 & 0 & -b \\
            0 & a - \lambda & 0 & 0 & -b \\
            0 & 0 & a - \lambda & 0 & -b \\
            0 & 0 & 0 & a - \lambda & -b \\
            b & b & b & b & a - \lambda \\
        \end{vmatrix} = \\
        &= b {\left(a - \lambda\right)}^3 \left(-b\right) + \left(a - \lambda\right)\begin{vmatrix}
            a - \lambda & 0 & 0 & -b \\
            0 & a - \lambda & 0 & -b \\
            0 & 0 & a - \lambda & -b \\
            b & b & b & a - \lambda \\
        \end{vmatrix} = \\
        &= b {\left(a - \lambda\right)}^3 \left(-b\right) + \left(a - \lambda\right)\left(b{\left(a-\lambda\right)}^2\left(-b\right)
        + \left(a-\lambda\right)\left(b\left(a-\lambda\right)\left(-b\right) + \left(a-\lambda\right)\left({\left(a\lambda\right)}^2 + b^2\right)\right)\right) = \\
        &= -{\left(a - \lambda\right)}^3\left(b^2 + \left(b^2 + \left(b^2 + \left({\left(a - \lambda\right)}^2 + b^2\right)\right)\right)\right) = \\
        &= {\left(\lambda - a\right)}^3\left(4b^2 + {\left(a - \lambda\right)}^2\right) = \\
        &= {\left(\lambda - a\right)}^2\left(a - \lambda + 2bi\right)\left(a - \lambda - 2bi\right).
    \end{align*}
    Torej so lastne vrednosti matrike \(A\) enake \(a\) in \(\lambda = a \pm 2bi\).

    Iščemo \(a\) in \(b\), da bo \(A\) unitarna. Torej mora \(A\) ohranjati dolžine vektorjev, torej ohranja tudi dolžine enotskih baznih vektorjev:
    \[Ae_1 = \begin{bmatrix}
        a \\ 0 \\ 0 \\ 0 \\ b
    \end{bmatrix}, \dots, Ae_4 = \begin{bmatrix}
        0 \\ 0 \\ 0 \\ a \\ b
    \end{bmatrix}, Ae_5 = \begin{bmatrix}
        -b \\ -b \\ -b \\ -b \\ a
    \end{bmatrix}\]
    Iz prve (in do četrte) enačbe sledi: \({|a|^2} + {|b|}^2 = 1\) in iz pete: \({|a|^2} + 4{|b|}^2 = 1\). Dobimo, da je
    \(b = 0\) in \(|a| = 1\). Moramo še preveriti, da so ti pogoji zadostni (vemo da baznim vektorjem ohranja dolžino, ne pa vsem),
    torej da velja \(A^H A = I\). Iz \(b = 0\) ugotovimo, da je \(A\) diagonalna matrika.
    Potem mora veljati \(a \cdot \overline{a} = 1\), kar je enako pogoju \(|a| = 1\).

\section*{2.~naloga}
    Preslikava \(A\) je endomorfizem prostora \(\mathbb{C}^{10}\)
    in ima lastne vrednosti enake \(0\), \(1\) in \(2\). Maksimalno število linearno neodvisnih lastnih vektorjev je
    pet. Lastna vrednost \(1\) je ničla tretje stopnje minimalnega polinoma preslikave
    \(A\) in ničla četrte stopnje karakterističnega polinoma preslikave \(A\).
    Lastna vrednost \(2\) je ničla tretje stopnje v minimalnem in karakterističnem
    polinomu preslikave \(A\).
    Določi rang \(A\), minimalni in karakteristični polinom \(A\) in zapiši Jordanovo
    formo preslikave \(A\).

\subsection*{Rešitev}
    5 neodvisnih lastnih vektorjev pomeni, da ima \(J(A)\) pet celic. Vemo še:
    \[
        \Delta_A(\lambda) = {\left(\lambda - 1\right)}^4{\left(\lambda - 2\right)}^3\lambda^3 \quad \text{in} \quad
        m_A(\lambda) = {\left(\lambda - 1\right)}^3{\left(\lambda - 2\right)}^3\lambda^x
    \]
    Potenca v karakterističnem polinomu za lastno vrednost \(0\) nam pove, da je \(1 \le \dim\ker A \le 3\). \(\dim\ker A\) je
    enak številu celic za \(\lambda = 0\).
    Potence v karakterističnem polinomu nam povejo skupno velikost celic v Jordanovi formi, potence v minimalnem polinomu pa
    velikosti največjih celic. Za \(\lambda = 1\) je največja celica velikosti 3, torej ima še eno celico velikosti 1.
    Za \(\lambda = 2\) je največja celica velikosti 3, torej je to edina celica. Za \(\lambda = 0\) ostaneta še dve celici,
    skupaj sta veliki 3. Torej je ena velika \(x = 2\) in ena velika 1. Rang matrike je \(8\).
    \[
        J(A) = \begin{bmatrix}
            1 & 1 &   &   &   &   &   &   &   &   \\
              & 1 & 1 &   &   &   &   &   &   &   \\
              &   & 1 &   &   &   &   &   &   &   \\
              &   &   & 1 &   &   &   &   &   &   \\
              &   &   &   & 2 & 1 &   &   &   &   \\
              &   &   &   &   & 2 & 1 &   &   &   \\
              &   &   &   &   &   & 2 &   &   &   \\
              &   &   &   &   &   &   & 0 & 1 &   \\
              &   &   &   &   &   &   &   & 0 &   \\
              &   &   &   &   &   &   &   &   & 0 \\
        \end{bmatrix}.
    \]

\section*{3.~naloga}
    Naj bo \(\langle \cdot \; , \cdot \rangle\) takšen skalarni produkt na \(R_4\left[x\right]\), da je množica \(\{1, x, x^2, x^3, x^4\}\) ortonormirana baza.
    Določi kakšno ortogonalno bazo ortogonalnega komplementa prostora \(U\), če je
    \[U = \left\{p \in \R_4\left[x\right]; \quad p(1) = 0, \quad p(x) = p(-x) \quad \forall x \in \R\right\}.\]

\subsection*{Rešitev}
    Vsak \(p \in U\) ima ničelne koeficiente pri lihih potencah:
    \begin{align*}
        p(x) &= p(-x) \\
        \alpha x^4 + \beta x^3 + \gamma x^2 + \delta x + \varepsilon &= \alpha x^4 - \beta x^3 + \gamma x^2 - \delta x + \varepsilon \\
        2\beta x^3 + 2\delta x &= 0 \\
    \end{align*}
    Kar mora veljati za vsak \(x\). Torej je \(\beta = \delta = 0\). Naj bo \(p(x) = a x^4 + b x^2 + c \in U\). Uporabimo
    še pogoj \(p(1) = 0\) in dobimo \(a + b + c = 0\) in \(p(x) = a x^4 + b x^2 - a - b\).
    Iščemo ortogonalni komplement za \(U\), torej rešujemo enačbo \(\langle p(x), q(x)\rangle = 0\), kjer je \(q(x) = \alpha x^4 - \beta x^3 + \gamma x^2 - \delta x + \varepsilon\)
    in \(p\) je poljuben polinom iz \(U\). Dobimo:
    \begin{align*}
        \langle p(x), q(x)\rangle &= \langle a x^4 + b x^2 - a - b, \alpha x^4 - \beta x^3 + \gamma x^2 - \delta x + \varepsilon\rangle = \\
        &= a\left(\alpha\langle x^4, x^4\rangle + \beta\langle x^4, x^3\rangle, \dots\right) + b\left(\alpha\langle x^2, x^4\rangle + \beta\langle x^2, x^3\rangle + \dots\right) + \dots = 0 \\
        &= a\alpha + 0\beta + b\gamma + 0\delta + \left(-a-b\right)\varepsilon = \\
        &= a\alpha + b\gamma - \left(a+b\right)\varepsilon = \\
        &= a(\alpha - \varepsilon) + b(\gamma - \varepsilon) = 0 \\
    \end{align*}
    To mora veljati za vse \(a, b \in \R\), torej je \(\alpha = \gamma = \varepsilon\).
    Torej je ortogonalni komplement \(U\) enak prostoru vseh polinomov oblike \(q(x) = \alpha x^4 + \beta x^3 + \alpha x^2 + \delta x + \alpha\).

    Baza ortogonalnega komplementa je \(\{x, x^3, x^4 + x^2 + 1\}\). Ortonormiramo jo postopoma: prva dva elementa imata dolžino 1 in 
    sta medsebojno pravokotna. Tretji element ima dolžino \(\sqrt{3}\), zato ga delimo s \(\sqrt{3}\). Na prejšnja dva je pa že pravokoten.
    Dobljena baza je:
    \[\{x, x^3, \frac{1}{\sqrt{3}}x^4 + \frac{1}{\sqrt{3}}x^2 + \frac{1}{\sqrt{3}}\}.\]

\section*{4.~naloga}
    Na bo \(V\) realen vektorski prostor in \(A: V \to V\) linearna preslikava.
    \begin{enumerate}[label=(\alph*)]
        \item Pokaži, da velja \(A^* = -A\) natanko takrat, kadar za vsak vektor \(x \in V\) velja \(\langle Ax, x\rangle = 0\).
        \item Pokaži, da je tedaj preslikava \(A + I\) obrnljiva.
    \end{enumerate}

\subsection*{Rešitev}
    Implikacija v desno:
    \[\langle Ax, x\rangle = \langle x, A^*x\rangle = -\langle x, Ax\rangle = -\langle Ax, x\rangle\]
    enakost \(\langle Ax, x\rangle = -\langle Ax, x\rangle\) pomeni \(\langle Ax, x\rangle = 0\).
    Implikacija v levo:
    \begin{multline*}
        0 = \langle A(x + y), (x + y)\rangle = \langle Ax + Ay, x + y\rangle = \langle Ax, x\rangle + \langle Ax, y\rangle + \langle Ay, x\rangle + \langle Ay, y\rangle = \\
        = \langle Ax, y\rangle + \langle Ay, x\rangle = \langle Ax, y\rangle + \langle x, Ay\rangle = \langle Ax, y\rangle + \langle A^*x, y\rangle =
        \langle Ax + A^*x, y\rangle = \\ = \langle (A + A^*)x, y\rangle = \langle 0x, y\rangle
    \end{multline*}
    Zadnja enakost velja za vsaka \(x, y \in \R\), zato je \(A + A^* = 0\).

    Za drugo točko moramo preveriti, da \(-1\) ni lastna vrednost za \(A\). Recimo, da je \(-1\) lastna vrednost in \(x \ne 0\) njen lastni vektor.
    Potem velja \(0 = \langle Ax, x\rangle = \langle -x, x\rangle = -\langle x, x\rangle = 0\). Iz tega bi sledilo \(x = 0\), kar pa ni res. Torej \(-1\)
    ni lastna vrednost in \(\det(A + I) = \det(A - (-1) I) \ne 0\). Sledi, da je \(A + I\) obrnljiva.

\end{document}
