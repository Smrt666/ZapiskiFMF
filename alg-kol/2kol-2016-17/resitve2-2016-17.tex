\documentclass[a4,11pt]{article}

\usepackage{amsmath, amsthm, amssymb, amsfonts}
\usepackage[utf8]{inputenc}
\usepackage[T1]{fontenc}
\usepackage[slovene]{babel}
\usepackage{thmtools}
\usepackage{lmodern}
\usepackage[colorlinks=false]{hyperref}
\usepackage{textcomp}
\usepackage{enumitem}

\usepackage{floatrow}
\usepackage{wrapfig}

\usepackage{graphicx}
\usepackage{tikz}
\usepackage{tkz-euclide}
\usepackage{setspace}
\usepackage{geometry}
\usepackage{float}
\usepackage{framed}
\usepackage{mathrsfs}

\newcommand{\degree}{^\circ}
\newcommand{\R}{\mathbb{R}}
\DeclareMathOperator{\im}{im}
\DeclareMathOperator{\Lin}{Lin}

\title{Rešitve 2.~kolokvija iz algebre 2016/2017}
\author{}
\date{}

\begin{document}

\maketitle

\section*{1.~naloga}
    Dana je množica 
    \[G = \left\{\left(a, b, c\right) \in \R; ac \ne 0\right\}\]
    in pravilo 
    \[
        \left(a_1, b_1, c_1\right) \circ \left(a_2, b_2, c_2\right) = \left(a_1a_2, a_1b_2 + b_1c_2, c_1c_2\right).
    \]
    \begin{enumerate}[label=(\alph*)]
        \item Pokaži, da je pravilo \(\circ\) operacija in da je \(G\) za to operacijo grupa.
        \item Utemelji, da je podmnožica
            \[
                H = \left\{\left(a, b, c\right) \in G; a = c = 1\right\}
            \]
            podgrupa, ki je izomorfna grupi realnih števil za seštevanje.
    \end{enumerate}

\subsection*{Rešitev}
\begin{enumerate}[label=(\alph*)]
    \item Da je \(\circ\) operacija na \(G\) moramo preveriti, da je definirana za vse elemente iz \(G\)
    in da je za to operacijo zaprta.  \(\left(a_1, b_1, c_1\right) \circ \left(a_2, b_2, c_2\right) = \left(a_1a_2, a_1b_2 + b_1c_2, c_1c_2\right)\).
    Ker so \(a_1, a_2, b_1, b_2, c_1, c_2 \in \R\) je očitno, da je operacija definirana za vse elemente iz \(G\), vsi ti elementi imajo za člene
    realna števila, zato moramo preveriti le še pogoj za neničelnost produkta prvega in zadnjega člena.
    Vemo, da \(a_1c_1 \ne 0\) in \(a_2c_2 \ne 0\), torej so vsi \(a_1, a_2, c_1, c_2\) različni od \(0\). Posledično je tudi 
    \(a_1a_2\ne 0\) in \(c_1c_2 \ne 0\), torej je operacija zaprta za vse elemente iz \(G\).

    Preverimo asociativnost:
    \begin{align*}    
        & \left(\left(a_1, b_1, c_1\right) \circ \left(a_2, b_2, c_2\right)\right) \circ \left(a_3, b_3, c_3\right) =
        \left(a_1a_2, a_1b_2 + b_1c_2, c_1c_2\right) \circ \left(a_3, b_3, c_3\right) \\
        & = \left(a_1a_2a_3, a_1a_2b_3 + \left(a_1b_2 + b_1c_2\right)c_3, c_1c_2c_3\right) \\
        & = \left(a_1a_2a_3, a_1\left(a_2b_3 + b_2c_3\right) + b_1c_2c_3, c_1c_2c_3\right) \\
        & = \left(a_1, b_1, c_1\right) \circ \left(a_2a_3, a_2b_3 + b_2c_3, c_2c_3\right) \\
        & = \left(a_1, b_1, c_1\right) \circ \left(\left(a_2, b_2, c_2\right) \circ \left(a_3, b_3, c_3\right)\right).
    \end{align*}

    Enota za to operacijo je \(\left(1, 0, 1\right)\), inverz pa je \(\left(a, b, c\right)^{-1} = \left(\frac{1}{a}, -\frac{b}{ac}, \frac{1}{c}\right)\):
    \begin{align*}        
        \left(1, 0, 1\right) \circ \left(a, b, c\right) & = \left(1a, 1b + 0c, 1c\right) = \left(a, b, c\right) \\
        \left(a, b, c\right) \circ \left(1, 0, 1\right) & = \left(a1, a0 + b1, c1\right) = \left(a, b, c\right). \\
        \left(a, b, c\right) \circ \left(\frac{1}{a}, -\frac{b}{ac}, \frac{1}{c}\right) & = \left(a\frac{1}{a}, a\left(-\frac{b}{ac}\right) + b\frac{1}{c}, c\frac{1}{c}\right) = \left(1, 0, 1\right) \\
    \end{align*}
    Ker \(a, c \ne 0\), je inverz definiran za vse elemente iz \(G\). Sledi da je \(G\) grupa za operacijo \(\circ\).

    \item Elementi množice \(H\) so oblike \(\left(1, b, 1\right)\), kjer je \(b \in \R\).
    Da je \(H\) podgrupa grupe \(G\) moramo preveriti, da je zaprta za kompozitum z inverzom.
    Izberemo poljubna \(\left(1, x, 1\right), \left(1, y, 1\right) \in H\).
    Inverz elementa \(\left(1, y, 1\right)\) je \(\left(1, -y, 1\right)\).
    Izračunajmo \(\left(1, x, 1\right) \circ \left(1, -y, 1\right) = \left(1, -y + x, 1\right) \in H\).

    Da je \(H\) izomorfna grupi realnih števil za seštevanje, definiramo preslikavo
    \(\varphi: H \to \R; \varphi\left(1, b, 1\right) = b\).
    Preverimo, da je \(\varphi\) homomorfizem:
    \[
        \varphi\left(\left(1, x, 1\right) \circ \left(1, y, 1\right)\right)
        = \varphi\left(1, x + y, 1\right) = x + y 
        = \varphi\left(1, x, 1\right) + \varphi\left(1, y, 1\right).
    \]
    Preveriti moramo še za inverz \(\varphi^{-1}\): \(\varphi^{-1}\left(x\right) = \left(1, x, 1\right)\)
    \[
        \varphi^{-1}\left(x + y\right) = \left(1, x + y, 1\right) 
        = \left(1, x, 1\right) \circ \left(1, y, 1\right) 
        = \varphi^{-1}\left(x\right) \circ \varphi^{-1}\left(y\right).
    \]
\end{enumerate}

\section*{2.~naloga}
    Naj bo \(S_n\) grupa permutacij elementov \(\{1, 2, \ldots, n\}\).
    Naj bodo \(\left(ab\right)\), \(\left(bc\right)\) in \(\left(cd\right)\)
    transpozicije v \(S_n\).
    \begin{enumerate}[label=(\alph*)]
        \item Če je \(a \ne c\), pokaži, da se da produkt \(\left(ab\right)\left(bc\right)\)
        napisati kot cikel dolžine 3.
        \item  Če so \(a\), \(b\), \(c\) in \(d\) različna števila, pokaži, da se da produkt
        \(\left(ab\right)\left(cd\right)\) napisati kot produkt dveh ciklov dolžine 3.
        \item Pokaži, da se vsaka soda permutacija napisati kot produkt ciklov dolžine 3.
    \end{enumerate}

\subsection*{Rešitev}
\begin{enumerate}[label=(\alph*)]
    \item Produkt \(\left(ab\right)\left(bc\right)\) je enak \(\left(abc\right)\).
    \item Produkt \(\left(ab\right)\left(cd\right) = \left(dbc\right)\left(adb\right)\).
    \item Naj bo \(\sigma \in S_n\) soda permutacija. Ker je \(\sigma\) soda, je enak produktu sodega števila
    transpozicij. Okrog 1.~in 2.~transpozicije v produktu lahko napišemo oklepaje, 
    potem okrog 3.~in 4.~transpozicije, itd. Ker je transpozicij sodo, bodo 
    vse v parih. Če sta v paru dve transpoziciji, ki imata dva skupna elementa,
    je njun produkt enak identiteti. Identiteto lahko zapišemo kot produkt ciklov dolžine 3:
    \(\mathrm{id} = \left(1 2 3\right)\left(3 2 1\right)\). 
    
    Če imamo dve transpoziciji,
    ki se ujemata v enem elementu, lahko njun produkt zapišemo kot produkt dveh ciklov dolžine 3
    po pravilu iz točke (a). Pri tem se lahko zgodi, da cikla nista zapisana v obliki, 
    kjer bi imela ujemajoči se element na notranji strani. V tem primeru lahko upoštevamo
    \(\left(ab\right) = \left(ba\right)\) in prepišemo transpoziciji v želeno obliko.

    Če imamo dve transpoziciji, ki nimata skupnih elementov, lahko njun produkt zapišemo kot produkt
    dveh ciklov dolžine 3 po pravilu iz točke (b).

    Sledi, da lahko produkt dveh transpozicij zapišemo kot produkt ciklov dolžine 3.
    Torej lahko poljubno sodo permutacijo zapišemo kot produkt ciklov dolžine 3.
\end{enumerate}

\section*{3.~naloga}
    Najprej določi realni števili \(\alpha\) in \(\beta\), tako da bosta množici
    \[
        U = \left\{p \in \R_3\left[x\right]; p\left(0\right) - \left(\alpha - 1\right)
        p'\left(0\right) + \alpha = \beta\right\}
    \]
    \[
        \text{in } V = \left\{p \in \R_3\left[x\right]; p\left(0\right) - p''\left(x\right)
        = \beta + 1, p\left(-1\right) = 0\right\}
    \]
    vektorska podprostora prostora \(\R_3\left[x\right]\) realnih polinomov stopnje največ 3.
    Nato določi kakšne baze prostorov \(U\), \(V\) in \(U \cap V\).

\subsection*{Rešitev}
    Naj bo \(p\left(x\right) = ax^3 + bx^2 + cx + d \in U\).
    Iz pogoja \(p\left(0\right) - \left(\alpha - 1\right)p'\left(0\right) + \alpha = \beta\) sledi:
    \[
        d - \left(\alpha - 1\right)c + \alpha = \beta
    \]
    Ker vsak podprostor vsebuje \(0\), mora veljati enačba veljati za \(a = b = c = d = 0\).
    Dobimo \(\alpha = \beta\). Vstavimo nazaj v enačbo in dobimo \(d = \left(\alpha - 1\right)c\).
    To so vsi pogoji za podprostor \(U\), saj z linearnimi kombinacijami teh polinomov dobimo
    polinome, ki zadoščajo enačbi --- pri množenju s skalarjem se enačba pomnoži s skalarjem, 
    pri seštevanju se pa dve ekvivalentni enačbi seštejeta.

    Polinomi podprostora \(V\) so oblike 
    \[
        p\left(x\right) = \left(x + 1\right)\left(ax^2 + bx + c\right) 
        = ax^3 + \left(a + b\right)x^2 + \left(b + c\right)x + c.
    \]
    Iz pogoja \(p\left(0\right) - p''\left(x\right) = \beta + 1\) sledi:
    \(c - 6 a x - 2\left(a + b\right) = \beta + 1\). To mora veljati za vsak \(x\),
    zato je \(a = 0\). Tudi v tem prostoru mora biti polinom \(0\), zato mora veljati
    tudi \(0 = \beta + 1\) oziroma \(\beta = -1\). Torej v podprostoru \(V\)
    polinomi izpolnjujejo pogoj kadar je \(c = 6ax + 2a + 2b = 2b\).

    Zdaj vemo, da je \(\alpha = \beta = -1\). S tem kar smo do zdaj ugotovili,
    lahko zapišemo vse polinome iz \(U\) kot \(p\left(x\right) = ax^3 + bx^2 + cx - 2c\).
    Baza prostora \(U\) je torej \(\left\{x^3, x^2, x - 2\right\}\).
    Polinomi iz \(V\) so oblike 
    \(p\left(x\right) = \left(x + 1\right)\left(ax^2 + bx + c\right) = bx^2 + 3bx + 2b\).
    Baza prostora \(V\) je torej \(\left\{x^2 + 3x + 2\right\}\). Ker je \(V\) enorazsežen,
    imata \(U\) in \(V\) lahko ali trivialen presek ali pa je presek enak \(V\).
    V primeru netrivialnega preseka lahko zapišemo \(x^2 + 3x + 2\) kot linearno kombinacijo
    \(x^3\), \(x^2\), \(x - 2\), kar pa ni mogoče. Torej je presek trivialen in nima baze.

\section*{4.~naloga}
    Dana je preslikava \(A: \R^3 \to \R^3\),
    \[
        A\left(x, y, z\right) = \left(x + z, 2y - z, x + 2y\right).
    \]
    \begin{enumerate}[label=(\alph*)]
        \item Pokaži, da je \(A\) linearna preslikava.
        \item Poišči jedro preslikave \(A\).
        \item Katere premice se s preslikavo \(A\) preslikajo v točke?
        \item Katere ravnine preslikava \(A\) preslika v premice?
    \end{enumerate}

\subsection*{Rešitev}
\begin{enumerate}[label=(\alph*)]
    \item Najprej preverimo linearnost: \begin{multline*}
        A\left(\lambda\left(x, y, z\right) + \mu\left(x', y', z'\right)\right) = A\left(\lambda x + \mu x', \lambda y + \mu y', \lambda z + \mu z'\right) = \\
        = \left(\lambda x + \mu x' + \lambda z + \mu z', 2\lambda y + 2\mu y' - \lambda z - \mu z', \lambda x + \mu x' + 2\lambda y + 2\mu y'\right) = \\
        = \lambda\left(x + z, 2y - z, x + 2y\right) + \mu\left(x' + z', 2y' - z', x' + 2y'\right) = \lambda A\left(x, y, z\right) + \mu A\left(x', y', z'\right).
    \end{multline*}
    \item Za jedro preslikave \(A\) mora veljati \(A\left(x, y, z\right) = \left(0, 0, 0\right)\).
        Dobimo sistem enačb:
        \begin{align*}
            x + z & = 0, \\
            2y - z & = 0, \\
            x + 2y & = 0.
        \end{align*}
        Napišemo matriko homogenega sistema:
        \[
            \begin{bmatrix}
                1 & 0 & 1 \\
                0 & 2 & -1 \\
                1 & 2 & 0
            \end{bmatrix}
            \sim
            \begin{bmatrix}
                1 & 0 & 1 \\
                0 & 2 & -1 \\
                0 & 2 & -1
            \end{bmatrix}
            \sim
            \begin{bmatrix}
                1 & 0 & 1 \\
                0 & 2 & -1 \\
                0 & 0 & 0
            \end{bmatrix}.
        \]
        Rešitev sistema je \(x = -z\), \(y = \frac{1}{2}z\), \(z = z\), torej je jedro preslikave enako \(\Lin\left\{(-1, \frac{1}{2}, 1)\right\}\). 
    \item Če ima premica smerni vektor \(\left(a, b, c\right)\), se slika v točko natanko tedaj,
        ko za vsako točko \(\left(x, y, z\right)\) na premici velja:
        \[
            A\left(\left(x, y, z\right) + \lambda\left(a, b, c\right)\right) = A\left(x, y, z\right)
        \]
        Ker je \(A\) linearna preslikava, lahko to prepišemo kot:
        \[
            \lambda \left(a, b, c\right) \in \ker A.
        \]
        Torej se v točke slikajo natanko premice, ki imajo smerni vektor v jedru preslikave \(A\)
        --- premice vzporedne vektorju \((-1, \frac{1}{2}, 1)\).

    
    \item Če se ravnina \(\Pi\) slika v premico, potem se tudi ta ravnina premaknjena
        za poljuben vektor \(\left(a, b, c\right)\) slika v premico. To najlažje vidimo tako, da
        pogledamo sliko ravnine \(\Pi\), ki je premica. Vsak vektor \(\left(x, y, z\right) \in \Pi\) se slika
        v točko na premici. Ker je \(A\) linearna, se vsak vektor \(\left(x, y, z\right) + \left(a, b, c\right)\) slika v 
        isto točko na premici premaknjeno za \(A\left(a, b, c\right)\). Torej vzporeden premik 
        ravnine povzroči vzporeden premik slike, če je ta slika premica, ostane premica.
        Zato je potrebno pogledati le ravnine, ki grejo skozi izhodišče.

        Ravnina \(\Pi\), ki gre skozi izhodišče, je podprostor.
        Dimenzija ravnine je \(2\) in za \(A\) zoženo na \(\Pi\) velja \(\dim \Pi = \dim \ker A + \dim \im A\).
        Iščemo ravnine, ki se slika v premice, torej mora veljati \(\dim \im A = 1\).
        Sledi, da je \(\dim \ker A = 1\). Torej mora imeti \(A\) netrivialno jedro v \(\Pi\).
        Jedro preslikave \(A\) v \(\R^3\) ima eno dimenzijo, torej je to isto jedro tudi jedro 
        preslikave \(A\) v \(\Pi\). Torej ravnine skozi izhodišče, ki se slikajo v premice,
        vsebujejo jedro preslikave \(A\).

        Velja tudi obratno, vsaka ravnina skozi izhodišče, ki vsebuje jedro preslikave \(A\),
        se slika v premico. To spet sledi iz \(\dim \Pi = \dim \ker A + \dim \im A\), saj v 
        tem primeru predpostavimo, da je \(\dim \ker A = 1\). Ker je \(\dim \Pi = 2\) je potem
        \(\dim \im A = 1\), torej je slika ravnine \(\Pi\) premica.

        Vse to skupaj pomeni, da se v premice slikajo ravnine, ki so vzporedne ravninam,
        ki vsebujejo jedro preslikave \(A\). Z drugimi besedami, ravnine, ki se slikajo v premice,
        so natanko ravnine vzporedne vektorju \((-1, \frac{1}{2}, 1)\).
\end{enumerate}

\end{document}