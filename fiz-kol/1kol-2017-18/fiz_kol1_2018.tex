\documentclass[a4,11pt]{article}

\usepackage{amsmath, amsthm, amssymb, amsfonts}
\usepackage[utf8]{inputenc}
\usepackage[T1]{fontenc}
\usepackage[slovene]{babel}
\usepackage{lmodern}
\usepackage[colorlinks=false]{hyperref}
\usepackage{textcomp}
\usepackage{enumitem}

\usepackage{mathrsfs}

\usepackage{siunitx}


\title{Rešitve 1.~kolokvija iz fizike 2017/2018}
\author{}
\date{}

\begin{document}

\maketitle

\section{Naloga}
    Voziček z maso \(m_v = \qty{100}{\kg}\) se giblje s hitrostjo \(v_v = \qty{1}{\m\per\s}\) po vodoravnem tiru. Človek z maso \(m_c = \qty{50}{\kg}\)
    priteče pod kotom \(\varphi = \ang{60}\) proti tiru in skoči na voziček. Po doskoku voziček, ki ostane na tiru,
    in človek na njem obmirujeta. S kolikšno hitrostjo je tekel človek?

\subsection*{Rešitev}
    Pogledamo gibalne količine v smeri tira.
    V smeri tira deluje sila le med človekom in vozičkom. Če vzamemo za sistem človeka in voziček, se sunka sil odštejeta
    in gibalna količina sistema se ohranja. Telesi obmirujeta (\(G = 0\)). Torej velja:
    \[m_v v_v - m_c v_c \cos\varphi = 0.\]
    Iz tega izraza izrazimo hitrost človeka:
    \[v_c = \frac{m_v v_v}{m_c \cos\varphi} = \qty{4}{\m\per\s}.\]

\section{Naloga}
    Klado z maso \(m = \qty{1}{\kg}\) potiskamo po klancu navzdol s silo \(F = \qty{5}{\N}\). Kolikšen naj bo kot \(\alpha\) med smerjo
    sile in klancem, da bo klada drsela navzdol z maksimalnim pospeškom? Koeficient trenja
    med klado in podlago je \(k = 0.2\), naklon klanca pa \(\varphi = \ang{30}\).

\subsection*{Rešitev}
    Najprej izračunamo silo trenja:
    \[F_t = k F_{\text{pravokotna na klanec}} = k \left(m g \cos\varphi - F \sin \alpha\right).\]
    Vsota sil na klado v smeri klanca je:
    \[ma = F_g \sin\varphi + F \cos\alpha - F_t = mg\sin\varphi + F\cos\alpha - k \left(m g \cos\varphi - F \sin \alpha\right)\]
    Pospešek ima maksimum, ko je odvod po kotu \(\alpha\) enak nič:
    \begin{align*}
        \frac{d}{d\alpha} \left(mg\sin\varphi + F\cos\alpha - k \left(m g \cos\varphi - F \sin \alpha\right)\right) &= 0, \\
        -F\sin\alpha + kF\cos\alpha &= 0, \\
        \tan\alpha &= k.
    \end{align*}
    Lahko preverimo, da je to res maksimum, tako da izračunamo drugi odvod \(0 < \alpha < \frac{\pi}{2}\):
    \[\frac{d^2}{d\alpha^2} a = -F\cos\alpha - kF\sin\alpha < 0.\]
    Torej je pospešek maksimalen, ko je kot med smerjo sile in klanca enak kotu trenja:
    \[\alpha = \arctan k = \ang{11.31}.\]

\section{Naloga}
    Okrogla plošča, ki na začetku miruje, se zavrti s kotnim pospeškom \(\alpha\left(t\right) = \alpha_0 \exp{\left(-\frac{t}{t_0}\right)}\),
    kjer sta \(\alpha_0 = \qty{1}{\per\s\squared}\)
    in \(t_0 = \qty{5}{\s}\), okoli osi, ki gre skozi središče plošče in je nanjo pravokotna.
    Kolikšna je kotna hitrost plošče po dolgem času? Ob katerem času bo pospešek točke na
    obodu plošče kazal pod kotom \ang{45} glede na zveznico med točko in središčem plošče?

\subsection*{Rešitev}
    Kotna hitrost plošče je:
    \begin{align*}
        d\omega &= \alpha\left(t\right) dt = \alpha_0 e^{-\frac{t}{t_0}} dt \\
        \int_0^\omega d\omega &= \alpha_0 \int_0^t e^{-\frac{t}{t_0}} dt \\
        \omega &= \alpha_0 t_0 - \alpha_0 t_0e^{-\frac{t}{t_0}}
    \end{align*}
    Za dolg čas \(t \to \infty\) je kotna hitrost:
    \[\omega = \alpha_0 t_0 = \qty{5}{\per\s}.\]

    Pospešek točke na obodu plošče bo kazal pod kotom \ang{45} ko bosta tangencialni in radialni komponenti pospeška enaki:
    \begin{align*}
        a_c &= a_t \\
        \omega^2 r &= \alpha r \\
        {\left(\alpha_0 t_0 - \alpha_0 t_0e^{-\frac{t}{t_0}}\right)}^2 &= \alpha_0 e^{-\frac{t}{t_0}} \\
        \alpha_0^2 t_0^2 - 2\alpha_0^2 t_0^2 e^{-\frac{t}{t_0}} + \alpha_0^2 t_0^2 e^{-\frac{2t}{t_0}} &= \alpha_0 e^{-\frac{t}{t_0}} \\
        e^{-\frac{t}{t_0}} &= \frac{-2\alpha_0^2t_0^2 - \alpha_0 \pm \sqrt{{\left(2\alpha_0^2t_0^2 + \alpha_0\right)}^2 - 4\alpha_0^4t_0^4}}{-2\alpha_0^2t_0^2} \\
    \end{align*}
    \[
        e^{-\frac{t}{t_0}} = 1 + \frac{1 \mp \sqrt{4\alpha_0t_0^2 + 1}}{2\alpha_0t_0^2}
    \]
    Zanima nas rešitev za \(t > 0\), torej \(e^{-\frac{t}{t_0}} < 1\). To zagotovo ne velja za rešitev s plusom pred korenom, zato vzamemo minus:
    \[
        t = - t_0 \ln\left(1 + \frac{1 - \sqrt{4\alpha_0t_0^2 + 1}}{2\alpha_0t_0^2}\right) = \qty{1}{\s}.
    \]

\section{Naloga}
    Raketo izstrelimo s površja Zemlje v smeri navpično navzgor. Ob vžigu motorjev je skupna
    masa rakete in goriva \(m_0\), masa goriva pa \(m_G\). Hitrost izpušnih plinov glede na raketo je \(v_i\),
    masni pretok goriva v motorje pa \(\phi_m\). Po kolikšnem času po vžigu motorjev se raketa odlepi
    od tal? Kolikšna je hitrost rakete v trenutku, ko zmanjka goriva? Spreminjanje težnega
    pospeška z višino nad površjem Zemlje zanemari.

\subsection*{Rešitev}
    Silo motorja na raketo zapišemo kot:
    \[F_m = \phi_m v_i.\]
    Masa rakete se zmanjšuje s hitrostjo \(\phi_m\), torej \(m = m_0 - \phi_m t\).
    Ko se raketa odlepi od tal, je sila motorja enaka gravitacijski sili:
    \begin{align*}
        F_m &= m g, \\
        \phi_m v_i &= \left(m_0 - \phi_m t_0\right) g, \\
        t_0 &= \frac{m_0 g - v_i\phi_m}{\phi_m g}.
    \end{align*}
    Če je sila motorja že od začetka večja od gravitacijske sile, bi dobili \(t_0 < 0\), v tem primeru za čas vzamemo \(t_0 = 0\).
    Če je \(\phi v_i \le \left(m_0 - m_G\right) g\), se raketa ne odlepi od tal.

    Izračunamo pospešek rakete in iz tega izpeljemo hitrost:
    \begin{align*}
        a &= \frac{F_m - F_g}{m} = \frac{\phi_m v_i}{m_0 - \phi_m t} - g \\
        dv &= \left(\frac{\phi_m v_i}{m_0 - \phi_m t} - g\right) dt \\
        \int_0^v dv &= \int_{t_0}^{t_G} \left(\frac{\phi_m v_i}{m_0 - \phi_m t} - g\right) dt \\
        v &= -v_i \ln\left(\frac{m_0 - \phi_m t_0}{m_0 - \phi_m t_G}\right) + g\left(t_0 - t_G\right) \\
    \end{align*}
    Izračunati je treba še \(t_G\), čas, ko zmanjka goriva:
    \(\phi_m t_G = m_G\) oz.~\(t_G = \frac{m_G}{\phi_m}\).
    Ko zmanjka goriva, je hitrost rakete:
    \[v = -v_i \ln\left(\frac{m_0 - \phi_m t_0}{m_0 - m_G}\right) + g\left(t_0 - \frac{m_G}{\phi_m}\right).\]

\end{document}