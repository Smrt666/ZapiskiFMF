\documentclass[a4,11pt]{article}

\usepackage{amsmath, amsthm, amssymb, amsfonts}
\usepackage[utf8]{inputenc}
\usepackage[T1]{fontenc}
\usepackage[slovene]{babel}
\usepackage{lmodern}
\usepackage[colorlinks=false]{hyperref}
\usepackage{textcomp}
\usepackage{enumitem}

\usepackage{mathrsfs}

\usepackage{siunitx}


\title{Rešitve 1.~kolokvija iz fizike 2022/2023}
\author{}
\date{}

\begin{document}

\maketitle

\section{Naloga}
    Vrtiljak se vrti enakomerno pojemajoče s kotnim pojemkom \(\alpha = \qty{0.1}{\per\s\squared}\)
    in se po \(t_0 = \qty{20}{\s}\) ustavi.
    Kolikšen je bil na začetku ustavljanja radialni pospešek sedežev, ki so \(r = \qty{2}{\m}\) od osi vrtenja?
    Po koliko obratih se vrtiljak ustavi?

\subsection*{Rešitev}
    Radialni pospešek izračunamo po formuli:
    \[a_{r0} = \omega_0^2 r\]
    kjer je \(\omega_0\) začetna kotna hitrost, ki jo izračunamo iz enačbe:
    \[0 = \omega_{t_0} = \omega_0 - \alpha t_0\]
    torej je \(a_{r0} = {\left(\alpha t_0\right)}^2 r = \qty{8}{\m\per\s\squared}\).

    Med ustavljanjem se vrtiljak obrne za \(\varphi = \omega_0 t_0 - \frac{1}{2}\alpha t_0^2 = \frac{1}{2}\alpha t_0^2\).
    Pri tem naredi \(n = \frac{\varphi}{2\pi} = 3.18\) obratov.

\section{Naloga}
    Na klancu z naklonom \(\varphi = \ang{10}\) miruje lesena klada z maso \(M = \qty{1}{\kg}\). V klado izstrelimo naboj z maso
    \(m = \qty{1}{g}\) v smeri vzporedno s klancem. Ta v kladi obstane, klada pa se nato premakne \(x = \qty{10}{\cm}\)
    po klancu navzgor in se tam ustavi. Kolikšna je bila hitrost naboja? Za koliko bi se klada
    premaknila, če bi naboj vanjo izstrelili v nasprotni smeri? Koeficient trenja med klado in
    podlago je \(k = 0.3\).

\subsection*{Rešitev}
    Najprej izračunamo silo trenja: \(F_t = k(M + m)g\cos\varphi\).
    V smeri gibanja delujta dinamična komponenta sile teže in sila trenja:
    \[(M + m) a = F_d + F_t = (M + m)g\sin\varphi + k(M + m)g\cos\varphi\]
    kjer je \(a\) pospešek klade. Iz enačbe za gibanje dobimo začetno hitrost klade:
    \[0 = v_0^2 - 2ax.\]
    Iz ohranitve gibalne količine sistema klade in naboja dobimo hitrost naboja:
    \[v_0 (m + M) = m v.\]
    Hitrost naboja je torej:
    \[v = v_0 \frac{m + M}{m} = \sqrt{2\frac{(M + m)g\sin\varphi + k(M + m)g\cos\varphi}{M + m}x} \frac{m + M}{m}\]
    oziroma \(\qty{960}{\m\per\s}\).

    Če bi naboj izstrelili v nasprotni smeri, bi se glede na smer gibanja spremenila samo smer dinamične komponente sile teže:
    \[(M + m) a_2 = - (M + m)g\sin\varphi + k(M + m)g\cos\varphi.\]
    Potem je \(a_2 = \qty{1.19}{\m\per\s}\).
    \[x_2 = \frac{v_0^2}{2a_2} = \qty{38.5}{\cm}.\]

\section{Naloga}
    S kolikšno silo \(F\) moramo potiskati klado z maso \(M\) v vodoravni smeri, da klade z masami
    \(M\), \(m_1\) in \(m_2\) ena glede na drugo mirujejo? Vrvica in škripec sta lahka, trenja ni.

\subsection*{Rešitev}
    Edini način, da \(m_2\) miruje, je da \(m_1\) nanjo deluje s silo enako teži \(m_2\):
    \[m_2 g = m_1 a.\]
    Potem bo mirovala tudi \(m_1\). Pospešujemo celoten sistem, zato je:
    \[(M + m_1 + m_2) a = F.\]
    Iz zgornjih enačb sledi:
    \[F = (M + m_1 + m_2) \frac{m_2}{m_1} g.\]

\section{Naloga}
    Homogena gibka veriga z dolžino \(l\) in maso \(m\) leži iztegnjena na vodoravni gladki mizi tako,
    da je del visi čez rob mize in se z enim koncem dotika tal. Višina mize nad tlemi je \(h < l\).
    Verigo, ki je sprva mirovala, izpustimo. S kolikšno hitrostjo zdrkne njen drugi konec preko
    roba mize? Kako se rezultat spremeni, če je miza hrapava, kar opisuje koeficient trenja \(k_t\)?
    (Miza je dovolj visoka, da veriga spolzi z nje.)

\subsection*{Rešitev}
    Z \(x\) označimo za koliko se je veriga že premaknila.
    Viseči del mase bo verigo pospeševal. Dokler ne zdrkne konec z mize, bo ta masa \(\frac{h}{l}m\).
    Pospeševal se bo del verige, ki še ni na tleh, ta del ima maso \(\frac{l - x}{l}m\).
    Pospešek verige je:
    \begin{align*}
        a &= g\frac{h}{l - x} & \\
        \frac{dv}{dt} &= g\frac{h}{l - x} & \big/ \cdot v dt = dx\\
        v dv &= g\frac{h}{l - x} dx & \\
        \int_0^v v dv &= \int_0^{l - h} g\frac{h}{l - x} dx & \\
        \frac{v^2}{2} &= g h \ln\left(\frac{l}{h}\right) & \\
        v &= \sqrt{2 g h \ln\left(\frac{l}{h}\right)}. &
    \end{align*}

    Če trenje ni zanemarljivo:
    \begin{align*}
        m_{\text{ne na tleh}} a &= F_{\text{g visečega dela}} - F_t & \\
        \frac{l - x}{l} a &= \frac{h}{l} g - k_t \frac{l - x - h}{l} g & \\
        \left(l - x\right)\frac{dv}{dt} &= h g - k_t \left(l - x - h\right) g & \big/ \cdot v dt = dx \\
        v dv &= \frac{h g - l g k_t + x g k_t + h g k_t}{l - x} dx & \\
        \int_0^v v dv &= \int_0^{l - h} \frac{h g - l g k_t + x g k_t + h g k_t}{l - x} dx & \\
        \frac{v^2}{2} &= \left(hg + hgk_t\right)\ln\frac{l}{h} - glk_t + ghk_t & \\
        v &= \sqrt{2\left(hg + hgk_t\right)\ln\frac{l}{h} - 2glk_t + 2ghk_t} &
    \end{align*}

\end{document}