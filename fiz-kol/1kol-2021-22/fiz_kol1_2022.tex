\documentclass[a4,11pt]{article}

\usepackage{amsmath, amsthm, amssymb, amsfonts}
\usepackage[utf8]{inputenc}
\usepackage[T1]{fontenc}
\usepackage[slovene]{babel}
\usepackage{lmodern}
\usepackage[colorlinks=false]{hyperref}
\usepackage{textcomp}
\usepackage{enumitem}

\usepackage{mathrsfs}

\usepackage{siunitx}


\title{Rešitve 1.~kolokvija iz fizike 2021/2022}
\author{}
\date{}

\begin{document}

\maketitle

\section{Naloga}
    Ko ventilator, ki se vrti s frekvenco \qty{10}{\Hz}, izklopimo, se začne vrteti enakomerno pojemajoče.
    Za prvi vrtljaj po izklopu ventilator porabi \qty{0.2}{\s}. Izračunaj kotni pojemek. Po koliko vrtljajih
    se ventilator ustavi?

\subsection*{Rešitev}
    Za enakomerno pospešeno kroženje velja enačba:
    \[\varphi = \omega_0 t + \frac{\alpha t^2}{2},\]
    kjer je \(\omega_0\) začetna kotna hitrost, \(\alpha\) kotni pospešek, \(t\) čas in \(\varphi\) opravljen kot.
    Iz enačbe lahko izrazimo kotni pospešek (en obrat ustreza kotu \(2\pi\)):
    \[\alpha = \frac{2\left(\varphi - \omega_0 t\right)}{t^2} = \frac{2\left(\varphi - 2\pi\nu_0 t\right)}{t^2} = \qty{-314}{\per\second\squared}.\]
    Kotni pojemek je nasprotno enak pospešku, torej \(\alpha_{pojemek} = \qty{314}{\per\second\squared}\).

    Za enakomerno pospešeno kroženje velja tudi enačba:
    \[\omega^2 = \omega_0^2 + 2\alpha \varphi.\]
    Za ustavitev ventilatorja mora veljati \(\omega = 0\). Število obratov je \(N = \frac{\varphi}{2\pi}\). Dobimo:
    \[N = -\frac{\omega_0^2}{4\pi\alpha} = 1.\]

\section{Naloga}
    Smučarski skakalec se z roba odskočne mize odrine s hitrostjo \(v_0 = \qty{100}{\kilo\metre\per\hour} = \qty{27.77}{\metre\per\second}\) poševno navzgor
    pod kotom \(\alpha = \ang{10}\) glede na vodoravnico. Odskočna miza brez skoka preide v doskočišče, ki je
    raven klanec z naklonom \(\beta = \ang{40}\). Koliko časa traja let skakalca? Izračunaj dolžino njegovega
    skoka. Le-ta se meri od roba odskočne mize vzdolž doskočišča. Zračni upor zanemari.

\subsection*{Rešitev}
    Skakalec se giblje po paraboli, navpična komponenta gibanja je enakomerno pospešeno, vodoravna pa je enakomerno gibanje.
    Koordinatni sistem postavimo tako, da je izhodišče v začetni točki skoka. Enačbi gibanja sta:
    \begin{align*}
        y &= v_{0y} t - \frac{g}{2}t^2, \\
        x &= v_{0x} t.
    \end{align*}
    Kjer sta \(v_{0y} = v_0 \sin\alpha\) in \(v_{0x} = v_0 \cos\alpha\).
    Klanec lahko opišemo z enačbo \(y = -\tan\beta x\). Iščemo presečišče parabole leta in klanca. Vstavimo enačbi za gibanje v enačbo za klanec:
    \[v_{0y} t - \frac{g}{2}t^2 = -\tan\beta v_{0x} t.\]
    Iz enačbe izrazimo čas leta:
    \[t = \frac{2v_0\left(\sin\alpha + \cos\alpha\tan\beta\right)}{g} = \qty{5.66}{\s}.\]

    Dolžino skoka izračunamo iz enačbe za vodoravno gibanje in prejšnjega rezultata:
    \[d = \frac{x}{\cos\beta} = \frac{v_0\cos\alpha t}{\cos\beta} = \qty{202}{\metre}.\]

\section{Naloga}
    Z vrha klanca spustimo klado. Ta se na tretjini dolžine klanca zaleti v enako, mirujočo klado
    in se z njo sprime. Sprimek se nato na dveh tretjinah dolžine klanca zaleti v tretjo mirujočo
    klado, enako prvima dvema, in se z njo spet sprime. Kolik²no je razmerje hitrosti sprimka
    vseh treh klad na dnu klanca in hitrosti, ki bi jo tam imela prva klada, ki bi po enakem
    klancu drsela brez trkov? Trenje zanemari.

\subsection*{Rešitev}
    Označimo maso posamezne klade z \(m\) in višino klanca z \(h\). Poglejmo, kaj se zgodi ob prvem trku.
    Tik pred trkom velja:
    \begin{align*}
        v_1 &= \sqrt{2g\frac{h}{3}} \ \text{ hitrost klade ko se spusti za tretjino klanca} \\
        G_1 &= m v_1 \ \text{ gibalna količina klade ko se spusti za tretjino klanca} \\
    \end{align*}
    Pri trku se ohrani gibalna količina in velja:
    \begin{align*}
        G_1' &= G_1 = 2m v_2 \ \text{ gibalna količina sprimka po prvem trku} \\
        v_1' &= \frac{v_1}{2} = \sqrt{\frac{gh}{6}}\ \text{ hitrost sprimka po prvem trku} \\
        W_{k1}' &= \frac{2m v_1'^2}{2} = \frac{m g h}{6}\ \text{ kinetična energija sprimka po prvem trku} \\
    \end{align*}
    Tik pred drugim trkom velja:
    \begin{align*}
        W_{k2} &= W_{k1}' + 2m g \frac{h}{3} \ \text{ kinetična energija sprimka pred drugim trkom} \\
        \frac{2mv_2^2}{2} &= \frac{1}{6}mgh + \frac{2}{3}mgh = \frac{5}{6}mgh \\
        v_2 &= \sqrt{\frac{5}{6}gh} \ \text{ hitrost sprimka ko se spusti za dve tretjini klanca} \\
        G_2 &= 2m v_2 \ \text{ gibalna količina sprimka ko se spusti za dve tretjini klanca} \\
    \end{align*}
    Pri trku se spet ohrani gibalna količina in velja (zdaj je sprimek iz 3 teles):
    \begin{align*}
        G_2' &= G_2 = 3m v_2' \ \text{ gibalna količina sprimka po drugem trku} \\
        2m \sqrt{\frac{5}{6}gh} &= 3m v_2' \\
        v_2' &= \sqrt{\frac{10}{27}gh} \ \text{ hitrost sprimka po drugem trku} \\
        W_{k2}' &= \frac{3m v_2'^2}{2} = \frac{5}{9}mgh \ \text{ kinetična energija sprimka po drugem trku} \\
    \end{align*}
    Sprimek samo še zdrsi po zadnji tretjini klanca:
    \begin{align*}
        W_{k3} &= W_{k2}' + 3m g \frac{h}{3} = \frac{5}{9}mgh + mgh = \frac{14}{9}mgh\\
        \frac{3mv_3^2}{2} &= \frac{14}{9}mgh \\
        v_3 &= \sqrt{\frac{28}{27}gh} \ \text{ hitrost sprimka na dnu klanca} \\
    \end{align*}

    Hitrost ene klade, ki bi drsela po klanca brez trkov, je:
    \[u = \sqrt{2gh}.\]
    Razmerje hitrosti sprimka in hitrosti ene klade je:
    \[\frac{v_3}{u} = \sqrt{\frac{\frac{28}{27}gh}{2gh}} = \sqrt{\frac{14}{27}}.\]

\section{Naloga}
    Raketa z maso \(m_0 = \qty{100}{\tonne}\) in prečnim presekom \(S = \qty{10}{\metre\squared}\), ki po breztežnem prostoru potuje s
    hitrostjo \(v_0 = \qty{1}{\kilo\metre\per\second}\), zaide v mirujoč oblak prahu z gostoto \(\rho = \qty{0.1}{\kg\per\metre\cubed}\). Prah se na raketo lepi
    tako, da se pri tem njen prečni presek ne povečuje. Koliko časa po tem, ko je raketa vstopila v
    oblak prahu, se njena hitrost zmanjša na polovico začetne? Kolikšen je takrat njen pospešek?

\subsection*{Rešitev --- lažja verzija}
    Ker oblak miruje, lahko gibalno količino opišemo na enak način ne glede na to, kolikšen del oblaka vzamemo za sistem.
    Gibalna količina se ohranja, če za sistem vzamemo ves prah (lahko tudi mirujočega zraven), ki se je na raketo nalepil.
    \[m_0 v_0 = m v\]

    Z \(dm\) označimo maso prahu, ki se je nalepil na raketo v času \(dt\).
    \[dm = \rho S v dt\]
    Uporabimo prejšnjo enačbo:
    \begin{align*}
        dm &= \rho S \frac{m_0 v_0}{m} dt \\
        m dm &= \rho S m_0 v_0 dt \\
        \int_{m_0}^{m} m dm &= \int_{0}^{t} \rho S m_0 v_0 dt \\
        \frac{m^2 - m_0^2}{2} &= \rho S m_0 v_0 t \\
        \frac{{\left(\frac{m_0 v_0}{v}\right)}^2 - m_0^2}{2} &= \rho S m_0 v_0 t \\
    \end{align*}
    Za hitrost \(v = \frac{v_0}{2}\) dobimo:
    \begin{align*}
        \frac{{\left(\frac{m_0 v_0}{v_0/2}\right)}^2 - m_0^2}{2} &= \rho S m_0 v_0 t \\
        \frac{4m_0^2 - m_0^2}{2} &= \rho S m_0 v_0 t \\
        t &= \frac{3m_0}{2\rho S v_0} = \qty{150}{\s}
    \end{align*}
    Da izračunamo pospešek, izrazimo hitrost:
    \begin{align*}
        v &= \frac{m_0 v_0}{\sqrt{2\rho S m_0 v_0 t + m_0^2}} \\
        a &= \frac{dv}{dt} = - m_0^2 v_0^2 \rho S {\left(2\rho S m_0 v_0 t + m_0^2\right)}^{-\frac{3}{2}} \\
    \end{align*}
    Za \(t = \qty{150}{\s}\) dobimo:
    \[a = \qty{-1.25}{\metre\per\second\squared}.\]

\subsection*{Rešitev --- težja verzija}
    Ponavadi se take naloge rešuje z uporabo gibalne količine na nekoliko drugačen način.
    Veljajo iste enačbe kot prej:
    \begin{align*}
        m_0 v_0 &= m v \\
        dm &= \rho S v dt \\
    \end{align*}
    Naprej pa rešujemo nalogo s pomočjo gibalne količine:
    \begin{align*}
        0 = dG &= v dm + m dv \ \text{ (odvod produkta) } \\
        v dm &= - m dv \\
        v dm &= -m dv \\
        v \rho S v dt &= -\frac{v_0 m_0}{v} dv \\
        \frac{\rho S}{v_0 m_0} dt &= - \frac{1}{v^3} dv \\
        \int_{0}^{t} \frac{\rho S}{v_0 m_0} dt &= - \int_{v_0}^{v} \frac{1}{v^3} dv \\
        \frac{\rho S}{v_0 m_0} t &= \frac{1}{2v^2} - \frac{1}{2v_0^2} \\
    \end{align*}
    Za hitrost \(v = \frac{v_0}{2}\) dobimo:
    \begin{align*}
        \frac{\rho S}{v_0 m_0} t &= \frac{1}{2{\left(\frac{v_0}{2}\right)}^2} - \frac{1}{2v_0^2} \\
        t &= \frac{3m_0}{2\rho S v_0} = \qty{150}{\s}
    \end{align*}

    Da izračunamo pospešek, izrazimo hitrost in dobimo isto kot prej:
    \begin{align*}
        v &= \frac{1}{\sqrt{\frac{2\rho S}{m_0 v_0}t + \frac{1}{v_0^2}}} = \frac{m_0 v_0}{\sqrt{2\rho S m_0 v_0 t + m_0^2}} \\
        a &= \frac{dv}{dt} = - m_0^2 v_0^2 \rho S {\left(2\rho S m_0 v_0 t + m_0^2\right)}^{-\frac{3}{2}} = \qty{-1.25}{\metre\per\second\squared} \\
    \end{align*}



\end{document}