\documentclass[12pt]{article}

\usepackage{amsmath,amssymb,amsthm}
\usepackage[utf8]{inputenc}
\usepackage[T1]{fontenc}
\usepackage[slovene]{babel}

\usepackage{tikz}
\usepackage{tkz-euclide}

\title{Zapiski s predavanj iz algebre 1}

\begin{document}
\maketitle

\section{Vektorji v \(\mathbb{R}^3\)}
\subsection{Koordinatni sistem}
    Model za realna števila je \textbf{realna os}: premica, ki na kateri označimo točki \(0\) -- izhodišče in \(1\) -- enoto.
    Običajno je \(1\) desno od \(0\). Na ta način smo v premico vpeljali koordinatni sistem. Vsakemu realnemu števil
    ustreza natanko ena točka na realni osi.

    Če je \(x > 0\), potem številu \(x\) ustreza točka desno od \(0\), ki je za \(x\) oddaljena od \(0\). Če je 
    \(x < 0\), številu \(x\) ustreza točka levo od \(0\), ki je za \(-x\) oddaljena od \(0\).
    Obratno tudi velja: Vsaki točki na realni osi ustreza natanko eno realno število.
    Imamo torej bijekcijo med \(\mathbb{R}\) in realno osjo.

    \[\mathbb{R}^2 = \mathbb{R} \times \mathbb{R} = \left\{\left(x, y\right); x \in \mathbb{R}, y \in \mathbb{R}\right\}\]
    Model za \(\mathbb{R}^2\) je ravnina z danim koordinatnim sistemom. Običajno uporabljamo pravokotni koordinatni sistem.
    Sestavljata ga pravokotni realni osi, ki se sekata v točki \(0\) -- izhodišču obeh realnih osi. Običajno je ena realna
    os vodoravna in kaže desno, druga pa navpična in kaže gor. Vodoravna se imenuje \textbf{abscisna} os, navpična pa
    \textbf{ordinatna} os.

    \begin{tikzpicture}
        % koordinatni sistem
        \draw[-stealth] (-3, 0) -- (3, 0);
        % \node at (3, 0) [below] {\(x\)};
        \draw[-stealth] (0, -3) -- (0, 3);
        % \node at (0, 3) [left] {\(y\)};

        % točka, vzporednici
        \coordinate[label = above right:$T$] (T) at (2, 2);
        \node at (T) [circle,fill,scale=0.2] {\(T\)};
        \draw[dashed] (2, 2) -- (2, 0);
        \draw[dashed] (2, 2) -- (0, 2);
        \node at (2, 0) [below] {\(x\)};
        \node at (0, 2) [left] {\(y\)};
    \end{tikzpicture}

    Izberimo točko \(T\) v ravnini. Vzporednici ordinatni osi, ki poteka skozi \(T\), seka abscisno os v natanko eni
    točko \(x\). Vzporednica abscisni osi, ki poteka skozi točko \(T\) seka ordinatno os v natanko eni točki \(y\).
    Točki \(T\) smo priredili urejen par števil \(\left(x, y\right) \in \mathbb{R}^2\). Točko označimo \(T\left(x, y\right)\)
    in pravimo, da sta \(x\) in \(y\) koordinati točke \(T\). To prirejanje je injektivno, saj če bi za eno točko našli
    več možnih koordinat, bi pomenilo, da je na več vzporednicah, kar pa ne more biti res. Velja tudi surjektivnost,
    iz tega dvojega pa sledi tudi bijektivnost.

    Dobili smo preslikavo iz ravnine v \(\mathbb{R}^2\), ki je bijekcija, zato bomo množico \(\mathbb{R}^2\) identificirali
    z ravnino z danim koordinatnim sistemom.

    \begin{tikzpicture}
        \coordinate[label = left:$T_1$] (T1) at (0, 0);
        \coordinate[label = above right:$T_2$] (T2) at (3, 2);
        \node at (T1) [circle,fill,scale=0.2] {\(T_1\)};
        \node at (T2) [circle,fill,scale=0.2] {\(T_2\)};
        \draw[dashed] (T1) -- (3, 0);
        \draw[dashed] (T2) -- (3, 0);
        \draw (T2) -- (T1);
        \node at (3, 1) [right] {\(y_2 - y_1\)}
        \node at (1.5, 0) [below] {\(x_2 - x_1\)}
    \end{tikzpicture}

    \textbf{Razdalja} med točkama \(T_1\left(x_1, y_1\right)\) in \(T_2\left(x_2, y_2\right)\) je definirana s predpisom:
    \[d\left(T_1, T_2\right) = \sqrt{\left(x_1 - x_2\right) ^ 2 + \left(y_1 - y_2\right)^2}\]

\end{document}