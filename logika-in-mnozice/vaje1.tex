\documentclass[12pt]{article}

\usepackage{amsmath,amssymb,amsthm}
\usepackage[utf8]{inputenc}
\usepackage[T1]{fontenc}
\usepackage[slovene]{babel}
\usepackage{lmodern}

\usepackage{mathtools}

\usepackage{tikz}
\usepackage{tkz-euclide}

\usepackage{float}
\usepackage{enumitem}

\newcommand{\naloga}{\section{Naloga}}
\newcommand{\inj}[2][\text{}]{\text{in}_{#2}^{#1}}

\title{Rešitve vaj 1 pri predmetu logika in množice}

\begin{document}
\maketitle

\naloga
    Definirajmo množice \(A = \left\{1, 2, 3\right\}\), \(B = \left\{\ensuremath\heartsuit, \ensuremath\diamondsuit\right\}\), \(C = \left\{\ensuremath\star\right\}\).
    \textbf{Zapišite kakšen element:}
\naloga
\naloga
\naloga
\naloga
\naloga
\naloga
\naloga
\naloga
\naloga
    \textbf{Poiščite izomorfizem med množico strogo naraščajočih zaporedij realnih števil in množico zaporedij pozitivnih realnih števil.}

    Naj bo množica strogo naraščajočih zaporedij realnih števil \(A\), množica zaporedij pozitivnih realnih števil pa \(B\).
    Definirajmo preslikavo \(f: A \rightarrow B\) s predpisom:
    \begin{equation*}
        f: a \mapsto \begin{cases}
            \left(e^{a_0}, a_1 - a_0, \dots, a_n  - a_{n - 1}, \dots\right) & a\ \text{neskončno zaporedje}\\
            \left(e^{a_0}, a_1 - a_0, \dots, a_n  - a_{n - 1}\right) & a\ \text{končno zaporedje}
        \end{cases}
    \end{equation*}
    Ker je \(a\) strogo naraščajoče zaporedje, je \(a_{n} - a_{n - 1} > 0\), poleg tega je \(e^{a_0}\) vedno pozitivno število, zato je 
    prirejeno zaporedje zaporedje pozitivnih realnih števil.
    Definirajmo še preslikavo \(g: B \rightarrow A\) s predpisom: 
    \begin{equation*}
        g: b \mapsto \begin{cases}
            \left(\ln b_0, b_1 + \ln b_0, \dots, \ln b_0 + \sum_{i=1}^{n}b_i, \dots\right) & b\ \text{neskončno zaporedje}\\
            \left(\ln b_0, b_1 + \ln b_0, \dots, \ln b_0 + \sum_{i=1}^{n}b_i\right) & b\ \text{končno zaporedje}
        \end{cases}
    \end{equation*}
    Dobljeno zaporedje \(g(b), b \in B\) je strogo naraščajoče, saj je vsak \(b_i > 0\), \(i\)-ti člen je pa za \(b_i\) večji od prejšnjega.
    Izračunajmo sedaj kompozitum \(f \circ g\). Želimo, da bo enak identitetni preslikavi na \(B\). Naj bo \(b \in B\). Recimo, da je \(b\) neskončno zaporedje:
    \begin{align*}
        & (f \circ g)(b) = f(g(b)) = f\left(\left(\ln b_0, b_1 + \ln b_0, \dots, \ln b_0 + \sum_{i=1}^{n}b_i, \dots\right)\right) = \\
        &= \left(e^{\ln b_0}, \left(\ln b_0 + b_1\right) - \ln b_0, \dots, \left(\ln b_0 + \sum_{i=1}^{n}b_i\right) - \left(\ln b_0 + \sum_{i=1}^{n-1}b_i\right), \dots\right) = \\
        &= \left(b_0, b_1, \dots, b_n, \dots\right) = b
    \end{align*}
    Enako velja tudi za končno zaporedje \(b\). Podobno izračunamo še \(g \circ f\). Naj bo \(a \in A\) neskončno zaporedje:
    \begin{align*}
        & (g \circ f)(a) = g(f(a)) = g\left(\left(e^{a_0}, a_1 - a_0, \dots, a_n  - a_{n - 1}, \dots\right)\right) = \\
        &= \left(\ln e^{a_0}, \left(a_1 - a_0\right) + \ln e^{a_0}, \dots, \ln e^{a_0} + \sum_{i=1}^{n}\left(a_i - a_{i - 1}\right), \dots\right) = \\
        &= \left(a_0, a_1, \dots, a_0 + (a_n - a_0), \dots\right) = \\
        &= \left(a_0, a_1, \dots, a_n, \dots\right) = a
    \end{align*}
    Dobili smo, da je \(f \circ g\) identiteta na \(A\). Enako velja tudi za končno zaporedje \(a\). Torej sta \(f\) in \(g\) bijekciji, ki sta med seboj inverzni,
    \(A\) in \(B\) pa sta izomorfni. 
    \\

    \textbf{Poiščite izomorfizem med premico in krožnico brez ene točke.}

    \begin{figure*}[h!]
        \centering
        \begin{tikzpicture}
            \draw (-5, 0) -- (5, 0);
            \draw (0,2) circle (2);
            \coordinate[label = below right:$K$] (K) at (1.7320508, 1);
            \node at (K) [circle,fill,scale=0.3] {};
            \draw (0.86602540, -0.5) -- (3, 3.19615242);
            \coordinate[label = below right:$P$] (P) at (1.1547005, 0);
            \node at (P) [circle,fill,scale=0.3] {};
        \end{tikzpicture}
    \end{figure*}
    Premik premice je izomorfizem, ki premico preslika v premico. Premico premaknemo tako, da se krožnice dotika v eni točki. Tej točki na premici recimo \(P_0\),
    tej točki na krožnici pa \(K_0\). Izberemo poljubno točko \(P\) na premici. Narišemo tangenti na krožnico skozi to točko. Ena tangenta je premica sama, druga pa se dotika
    krožnice v točki \(K\). Točka \(P\) naj se slika v točko \(K\). Izberimo poljubno točko \(K\) na krožnici. Narišemo tangento na krožnico skozi to točko. 
    Tangenta je lahko vzporedna premici, če je izbrana točka \(K_0\), ki jo že znamo slikati na premico, ali pa točka na nasprotni strani krožnice. Slednja točka
    naj se ne slika nikamor -- to naj bo točka, ki krožnici manjka. Za vse ostale točke \(K\), pa obstaja natanko ena točka \(P\), ki se po prej napisanem slika v \(K\).
    Če torej rečemo, da se še \(K\) slika v \(P\), imamo bijekcijo med premico in krožnico brez ene točke.

\naloga
    \textbf{Naj bodo \(A\), \(B\) in \(C\) množice. Pokaži da  velja:}
    \begin{enumerate}[label=(\alph*)]
        \item \(A + B \cong B + A\) \\
            Naj bo \(f : A + B \rightarrow B + A, \inj{1}(x) \mapsto \inj{2}(x), \inj{2}(x) \mapsto \inj{1}(x)\) in \(g : B + A \rightarrow A + B, \inj{1}(x) \mapsto \inj{2}(x), \inj{2}(x) \mapsto \inj{1}(x)\).

            Izračunajmo kompozitum \(f \circ g\) za poljuben \(x \in B + A\): če je \(x = \inj{2}(a), a \in A\), potem je 
            \(\left(f \circ g\right)(x) = f(g(\inj{2}(a))) = f(\inj{1}(a)) = \inj{2}(a) = x\), sicer je pa \(x = \inj{1}(b), b \in B\) in 
            \(\left(f \circ g\right)(x) = f(g(\inj{1}(b))) = f(\inj{2}(b)) = \inj{1}(b) = x\). 
            
            V drugo smer velja podobno: če je \(x = \inj{1}(a), a \in A\), potem je 
            \(\left(g \circ f\right)(x) = g\left(f\left(\inj{1}(a)\right)\right) = g\left(\inj{2}(a)\right) = \inj{1}(a) = x\) in če je \(x = \inj{2}(b), b \in B\), potem je
            \(\left(g \circ f\right)(x) = g\left(f\left(\inj{2}(b)\right)\right) = g\left(\inj{1}(b)\right) = \inj{2}(b) = x\). Torej sta \(A + B\) in \(B + A\) izomorfni.
        \item \(A + (B + C) \cong (A + B) + C\) \\
            Naj bo \(f : A + (B + C) \rightarrow (A + B) + C, \inj{1}(a) \mapsto \inj{1}(\inj{1}(a)),\allowbreak \inj{2}(\inj{1}(b)) \mapsto \inj{1}(\inj{2}(b)),\allowbreak \inj{2}(\inj{2}(c)) \mapsto \inj{2}(c)\).
    \end{enumerate}

\naloga

\end{document}
