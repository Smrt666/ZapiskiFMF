\documentclass[11pt]{article}

\usepackage{amsmath, amsthm, amssymb, amsfonts}
\usepackage[margin=3cm]{geometry}
\usepackage[utf8]{inputenc}
\usepackage[T1]{fontenc}
\usepackage[slovene]{babel}
\usepackage{thmtools}
\usepackage{lmodern}
\usepackage[colorlinks=false]{hyperref}
\usepackage{textcomp}
\usepackage{enumitem}


\title{Rešitve 3.\ domače naloge iz diskretne matematike}
\author{}
\date{}

\begin{document}

\maketitle

\section*{1.~naloga}
S pomočjo Ferrersovih diagramov pokažite, da je število sebi konjugiranih razčlenitev velikosti \(n\) enako številu razčlenitev velikosti \(n\) 
s samimi različnimi lihimi sumandi. Razčlenitev
je sebi konjugirana, če je enaka svoji konjugirani razčlenitvi.

\subsection*{Rešitev}
Iščemo bijekcijo med Ferrersovimi diagrami, ki so simetrični čez diagonalo in diagrami, ki imajo same različne lihe vrstice.
Najprej gledamo sebi konjugirane diagrame.
Hitro lahko opazimo, da je prvi stolpec največji in enako dolg kot prva vrstica. Če odstranimo prvo vrstico in prvi stolpec, 
dobimo nov Ferrersov diagram, ki je sebi konjugiran. Skupaj smo odstranili liho elementov. Če postopek ponavljamo na preostanku,
dobimo same lihe skupine. Odvzet stolpec je bil največji (ostali so krajši ali enako dolgi). Ko smo odvzeli še prvo vrstico,
smo skrajšali vse preostale stolpce. Zato dobimo različne lihe skupine. Te odvzete skupine lahko po vrsticah zložimo v nov diagram,
ki ustreza razčlenitvi z lihimi različnimi sumandi. Očitno je ta preslikava injektivna.

V drugo smer moramo premisliti, da je vsak diagram z lihimi različnimi sumandi lahko zgrajen na tak način.
Vzamemo prvo (največjo) vrstico in jo uporabimo za prvo vrstico in prvi stolpec. To lahko na natanko en način storimo tako,
da je diagram simetričen čez diagonalo. Postopek nadaljujemo z ostalimi vrsticami. Ker so vse vrstice lihe, bomo vedno dobili
simetričen diagram. Ker so vrstice vedno krajše za vsaj 2, bomo vedno dobili veljaven diagram. Preslikava je torej bijekcija.

\section*{2.~naloga}
Poiščite vse razčlenitve velikosti \(20\) s samimi različnimi lihimi sumandi. S pomočjo tega
poiščite še vse sebi konjugirane razčlenitve velikosti \(20\).

\subsection*{Rešitev}
Pri fiksnem največjem seštevancu se sprašujemo, kako lahko z manjšimi seštevanci vsoto dopolnimo do 20. Pri tem 
si lahko prihranimo nekaj dela, če ugotovimo, da seštevancev ne more biti liho.
\begin{align*}
    19 + 1 & \sim 10 + 2 + 1 + 1 + 1 + 1 + 1 + 1 + 1 + 1 \\
    17 + 3 & \sim 9 + 3 + 2 + 1 + 1 + 1 + 1 + 1 + 1 \\
    15 + 5 & \sim 8 + 4 + 2 + 2 + 1 + 1 + 1 + 1 \\
    13 + 7 & \sim 7 + 5 + 2 + 2 + 2 + 1 + 1 \\
    11 + 9 & \sim 6 + 6 + 2 + 2 + 2 + 2 \\
    11 + 5 + 3 + 1 & \sim 6 + 4 + 4 + 4 + 1 + 1 \\
    9 + 7 + 3 + 1 & \sim 5 + 5 + 4 + 4 + 2 \\
\end{align*}

\section*{3.~naloga}
S pomočjo Ferrersovih diagramov pokažite, da je število razčlenitev velikosti \(n\) z največ
\(m\) členi enako številu razčlenitev velikosti \(n + m(m + 1)/2\) z \(m\) različnimi členi. Bijekcijo
med Ferrersovimi diagrami ilustrirajte tudi na enem konkretnem primeru za \(n = 8\) in
\(m = 5\).

\subsection*{Rešitev}
Najprej naredimo injekcijo iz razčlenitev \(n\) z največ \(m\) členi v razčlenitve \(n + m(m + 1)/2\) z \(m\) različnimi členi.
Prvo vrstico podaljšamo za \(m\), drugo za \(m - 1\), \ldots, zadnjo za \(1\). Možne prazne vrstice na koncu prav tako podaljšujemo, skupaj jih mora biti \(m\).
Tako dobimo diagram, ki ima \(m\) različno dolgih vrstic. Očitno je ta preslikava injektivna.

Za preslikavo v drugo smer vzamemo diagram z \(m\) različno dolgimi vrsticami. Iz prve vrstice odstranimo \(m\) elementov, iz druge \(m - 1\), \ldots, iz zadnje \(1\).
Dobimo diagram z največ \(m\) vrsticami. Torej je bila začetna preslikava surjektivna in posledično je bijektivna.

\section*{4.~naloga}
Seštejte vrsti \begin{enumerate}[label=(\alph*)]
    \item \(\sum_{k=0}^{n}\binom{2n}{2k}\),
    \item \(\sum_{k=0}^{n}k\binom{n}{k}\).
\end{enumerate}

\subsection*{Rešitev}
\begin{enumerate}[label=(\alph*)]
    \item Je več načinov, kako rešiti to nalogo. Lahko opazimo, da vsak člen \(\binom{2n}{2k}\) predstavlja število
        podmnožic množice z \(2n\) elementi, ki imajo \(2k\) elementov. Torej iščemo število podmnožic sode moči, kar je točno polovica vseh podmnožic.
        
        Še ena možnost je, da uporabimo \(0 = {\left(1 - 1\right)}^{2n} = \sum_{k=0}^{2n}\binom{2n}{k}{\left(-1\right)}^k\). Iskana vsota so ravno
        vsi pozitivni členi. Ker je njihova vsota nasprotno enaka vsoti negativnih, je njihova vsota torej polovica vsote
        \({\left(1 + 1\right)}^{2n} = \sum_{k=0}^{2n}\binom{2n}{k}\). Rezultat je \(2^{2n-1}\)
    \item En način je odvajanje funkcije \(f(x) = {(1 + x)}^n\). \(f'(x) = n{(1 + x)}^{n-1}\). Vstavimo \(x = 1\), da dobimo
        iskano vsoto. Rezultat je \(n \cdot 2^{n-1}\). 
        
        Drug način je \(\sum_{k=0}^{n} k\binom{n}{k} = \sum_{k=0}^{n} \left(n\binom{n}{k} - \left(n - k\right)\binom{n}{k}\right)
        = n \sum_{k=0}^{n} \binom{n}{k} - \sum_{k=0}^{n} \left(n - k\right)\binom{n}{k}\). To lahko poenostavimo v enačbo:
        \[\sum_{k=0}^{n} k\binom{n}{k} = n 2 ^ n - \sum_{k=0}^{n} k\binom{n}{k}\]
        in dobimo enak rezultat. 
\end{enumerate}

\section*{5.~naloga}
V razvoju multinoma \({\left(3a - 4b + 2c\right)}^7\) poiščite koeficient pred členom \(a^3bc^3\).

\subsection*{Rešitev}
V razpisu potence \(3\)-krat izberemo člen \(3a\), \(1\)-krat člen \(-4b\) in \(3\)-krat člen \(2c\). To lahko naredimo na \(\binom{7}{3,1,3}\) načinov.
\[\binom{7}{3,1,3} \cdot 3^3a^3 \cdot (-4)b \cdot 2^3c^3 = -2^7 \cdot 3^3 \cdot 5 \cdot 7\]

\section*{6.~naloga}
Naj bosta \(k\) in \(n\) naravni števili. Mesto sestavlja pravokotna mreža cest; \(k\) cest poteka v
smeri sever-jug, \(n\) cest pa v smeri vzhod-zahod. Študent stanuje na skrajno jugozahodnem
koncu mesta, fakulteta pa je na skrajno severovzhodnem delu mesta.
\begin{enumerate}[label=(\alph*)]
    \item Na koliko načinov lahko študent pride od doma do fakultete po najkrajši poti (torej
    če vedno hodi le proti severu ali proti vzhodu)?
    \item Kaj pa, če mora po vsaki cesti v smeri sever-jug prehoditi vsaj del poti (ceste ne
    sme samo prečkati)?
\end{enumerate}
Vsako od nalog najprej rešite v splošnem, potem pa še izračunajte za primer \(n = 11\) in
\(k = 6\).

\subsection*{Rešitev}
\begin{enumerate}[label=(\alph*)]
    \item Študent se bo \(n-1\)-krat premaknil proti severu in \(k-1\)-krat proti vzhodu. To lahko naredi na \(\binom{n+k-2}{n-1}\) načinov.
    \item Proti severu se bo premaknil \(n-1\)-krat, to bo opravil v \(k\) delih. Iščemo torej kompozicije števila \(n-1\) na \(k\) delov.
        To lahko naredi na \(\binom{n-2}{k-1}\) načinov.
\end{enumerate}
Za primer \(n = 11\) in \(k = 6\) dobimo: \(\binom{11 + 6 - 2}{11 - 1} = 3003\) in \(\binom{11 - 2}{6 - 1} = 126\).

\section*{7.~naloga}
Na koliko načinov lahko razdelimo razdelimo \(11\) različnih kroglic v enake škatle tako, da
je v dveh škatlah po ena kroglica, v treh škatlah po dve kroglice in v eni škatli tri kroglice?

\subsection*{Rešitev}
Izbrati moramo po eno kroglico za prvi dve škatli, \dots To lahko storimo na \(\binom{11}{1, 1, 2, 2, 2, 3}\) načinov.

\section*{8.~naloga}
Na koliko načinov lahko razdelimo razdelimo \(n\) različnih kroglic v enake škatle tako, da
je v \(k_i\) škatlah natanko \(i\) kroglic za \(n = 1, 2, \ldots, n\)? Pri tem je \(\sum_{i=1}^{n}i \cdot k_i = n\).
Zgled: pri prejšnji nalogi je \(n_1 = 2\), \(n_2 = 3\) in \(n_3 = 1\).

\subsection*{Rešitev}
Izberemo \(k_1\) škatel, v katere bomo dali po eno kroglico, \(k_2\) škatel, v katere bomo dali po dve kroglici, \dots
To lahko storimo na \(\frac{n!}{{\left(1!\right)}^{k_1} {\left(2!\right)}^{k_2} \cdots {\left(n!\right)}^{k_n}}\) načinov.

\section*{9.~naloga}
Koliko je permutacij nad množico z \(n\) elementi, ki imajo \(k_i\) disjunktnih ciklov dolžine \(i\)
za \(i = 1, 2, \ldots, n\)? Pri tem je \(\sum_{i=1}^{n}i \cdot k_i = n\).

\subsection*{Rešitev}
Izberemo \(k_1\) elementov za prvi cikel, \(k_2\) elementov za drugi cikel, \dots
Vsaka od teh izbranih skupin lahko predstavlja \(\left(i - 1\right)!\) različnih ciklov, kjer je \(i\) dolžina cikla. 
Torej je rešitev naloge:
\[\frac{n!}{{\left(1!\right)}^{k_1} {\left(2!\right)}^{k_2} \cdots {\left(n!\right)}^{k_n}} 
{\left(\left(1 - 1\right)!\right)}^{k_1} {\left(\left(2 - 1\right)!\right)}^{k_2} \cdots {\left(\left(n - 1\right)!\right)}^{k_n}
= \frac{n!}{1^{k_1} 2^{k_2} \cdots n^{k_n}}\]


\section*{10.~naloga}
Na koliko načinov lahko permutiramo črke \(O\), \(B\), \(Z\), \(O\), \(R\), \(I\), \(L\), \(U\), \(N\), \(A\), \(S\), \(I\), \(J\), \(E\)
\begin{enumerate}[label=(\alph*)]
    \item brez dodatnih omejitev?
    \item tako, da je \(A\) vedno pred \(Z\)?
    \item tako, da ni dveh zaporednih \(O\)-jev?
    \item tako, da soglasniki nastopajo po abecednem vrstnem redu?
\end{enumerate}

\subsection*{Rešitev}
\begin{enumerate}[label=(\alph*)]
    \item Rešitev je \(\frac{14!}{2!2!}\) (\(O\) in \(I\) se ponovita).
    \item Glede na prejšnji primer je ravno pol možnosti: \(\frac{14!}{2!2!2}\).
    \item Izračunamo število vseh permutacij z dvema zaporednima \(O\)-jema in ga odštejemo od števila vseh permutacij. Rešitev je
        \(\frac{14!}{2!2!} - \frac{13!}{2!}\).
    \item Postavimo \(B\), \(Z\), \(R\), \(L\), \(N\), \(S\) in \(J\), nato pa na ostalih 8 mest vrinemo preostalih 7 črk. 
    Pri tem pazimo, da lahko oba \(O\)-ja in \(I\)-ja zamenjamo in se vrstni red ne spremeni. Rešitev je
        \[\binom{8 + 7 - 1}{7 - 1} \frac{7!}{2!2!}\]
\end{enumerate}

\end{document}