\documentclass[11pt]{article}

\usepackage{amsmath, amsthm, amssymb, amsfonts}
\usepackage[margin=3cm]{geometry}
\usepackage[utf8]{inputenc}
\usepackage[T1]{fontenc}
\usepackage[slovene]{babel}
\usepackage{thmtools}
\usepackage{lmodern}
\usepackage[colorlinks=false]{hyperref}
\usepackage{textcomp}
\usepackage{enumitem}


\title{Rešitve 4.\ domače naloge iz diskretne matematike}
\author{}
\date{}

\begin{document}

\maketitle

\section*{1.~naloga}
Pokažite spodnje enakosti s pomočjo kombinatoričnih dokazov.
\begin{enumerate}[label=(\alph*)]
    \item
    \(
        \binom{2n}{2} = 2\binom{n}{2} + n^2,
    \)
    \item
    \(
        k\binom{n}{k} = n\binom{n-1}{k-1}.
    \)
\end{enumerate}

\subsection*{Rešitev}
\begin{enumerate}[label=(\alph*)]
    \item Leva stran je enaka številu izbir dveh elementov iz \([n] + [n]\). To je enako, če izberemo dva elementa
        iz prve vsotne ali pa dva iz druge vsotne množice, za vsako lahko to storimo na \(\binom{n}{2}\) načinov.
        Druga možnost je, da izberemo iz vsake vsotne množice po eno, za kar imamo \(n^2\) načinov.
    \item Štejemo pare, kjer je prvi člen \(k\)-ta podmnožica v \([n]\), drugi člen pa nek element iz nje. 
        Take pare lahko dobimo tudi, če izberemo najprej drugi element, za prvega pa vzamemo \(\left(k-1\right)\)-podmnožico
        ostalih elementov in ji dodamo vnaprej izbranega.
\end{enumerate}

\section*{2.~naloga}
Pokažite, da velja (dokaz z indukcijo in kombinatoričen dokaz):
\begin{enumerate}[label=(\alph*)]
    \item \(S(n, 2) = 2^{n - 1} - 1\) \quad za \quad \(n \ge 1\),
    \item \(S(n, n-1) = \binom{n}{2}\) \quad za \quad \(n \ge 1\).
\end{enumerate}

\subsection*{Rešitev}
\begin{enumerate}[label=(\alph*)]
    \item Pri kombinatoričnem dokazu iščemo razdelitve \(n\)-množice na \(2\) neprazna dela.
        Lahko gledamo neurejene pare podmnožic in njihovih komplementov. Opazimo, da jih je ravno pol toliko,
        kot podmnožic \([n]\). Izvzeti moramo še par prazne in cele množice in to je želen rezultat.

        Baza indukcije je za \(n=1\) \(S\left(1, 2\right) = 0 = 2 ^ {1 - 1} - 1\). Predpostavimo formulo za \(n\) in dobimo
        \[S\left(n + 1, 2\right) = S(n, 1) + 2 S\left(n, 2\right) = 1 + 2 \left(2 ^ {n - 1} - 1\right) = 2 ^ {n} - 1\]
    \item Ko \(n\)-množico delimo na \(n-1\) nepraznih delov, se pravzaprav sprašujemo, katera dva elementa bosta skupaj,
        saj bodo ostali sami.

        Za \(n=1\) enakost drži. Predpostavimo za \(n\).
        \[S\left(n + 1, n\right) = S\left(n, n - 1\right) + n S\left(n, n\right) = 
        \binom{n}{2} + n = \frac{n\left(n - 1\right) + 2n}{2} = \binom{n + 1}{2}.\]
\end{enumerate}

\section*{3.~naloga}
Pokažite naslednjo enakost za Stirlingovimi števili 2.~vrste (kombinatoričen dokaz in dokaz
dokaz z indukcijo):
\[
    x^n = \sum_{k=0}^{n}S\left(n, k\right) x ^ {\underline{k}} \text{\quad za \quad} n \ge 0.
\]

\subsection*{Rešitev}
Gledamo funkcije iz \({[x]}^{[n]}\). Po eni strani jih je \(x^n\), po drugi strani pa lahko razdelimo
\([x]\) na razrede, kjer se v isto vrednost slikajo natanko tisti elementi, ki so v istem razredu.
Za vsako tako razdelitev na \(k\) razredov lahko dobimo \(x^{\underline{k}}\) funkcij --- za vsak razred izberemo nek element, ki se pri
tem ne smejo ponavljati. Število takih razdelitev je \(S\left(n, k\right)\) ko seštejemo za vse \(k\) dobimo
formulo. Sledi, da formula velja za \(x \in \mathbb{N}\). Ker imamo na levi in na desni strani polinoma v \(x\) stopnje \(n\)
in se ujemata v neskončno točkah, sta enaka.

Za \(n = 0\) formula deluje (tako pri \(x=0\) kot pri \(x \ne 0\)). Predpostavimo, da velja za \(n\).
\begin{align*}
    \begin{array}{l}
   \sum_{k=0}^{n+1} S( n+1,k) x^{\underline{k}}\\
   \end{array} & =\sum _{k=1}^{n}( S( n,k-1) \ +\ kS( n,k)) x^{\underline{k}} +S( n+1,0) +S( n+1,n+1) x^{\underline{n+1}}\\
    & =\sum_{k=0}^{n-1} S( n,k)\left(kx^{\underline{k}} +x^{\underline{k+1}}\right) +S( n,n) nx^{\underline{n}} +x^{\underline{n+1}}\\
    & =\sum_{k=0}^{n-1} S( n,k) x^{\underline{k}}( k+x-k) +x^{\underline{n}}( x-n+n)\\
    & =x\sum_{k=0}^{n-1} S( n,k) x^{\underline{k}} +x\cdot x^{\underline{n}} S( n,n)\\
    & =x\sum_{k=0}^{n} S( n,k) x^{\underline{k}} \\
    & = x^{n+1}
\end{align*}

\section*{4.~naloga}
Pridruženo Stirlingovo število reda \(r\), \(Sr(n, k)\), je število razdelitev \([n]\) na \(k\) blokov, pri čemer
vsak blok vsebuje vsaj \(r\) elementov. Privzamemo, da velja \(n, k, r \ge 1\).
\begin{enumerate}[label=(\alph*)]
    \item Določite \(S_r\left(n, 1\right)\) za \(n \ge 1\). Za katere \(k\) je \(S_r\left(n, k\right) = 0\) (v odvisnosti os \(r\), \(n\))?
    \item Poiščite rekurzivno zvezo, ki ji zadoščajo števila \(S_r\left(n, k\right)\) za \(n, k \ge 2\) (kombinatoričen dokaz).
    \item S pomočjo rekurzivne zveze izračunajte \(S_3\left(10, 3\right)\).
\end{enumerate}

\subsection*{Rešitev}
\begin{enumerate}[label=(\alph*)]
    \item En blok z vsaj \(r\) elementi lahko naredimo na enega ali pa \(0\) načinov. 
    \[S_r\left(n, 1\right) = \begin{cases}
        1, & n \ge r, \\
        0, & \text{sicer}
    \end{cases}.\]
    Hitro lahko ugotovimo, da je \(S_r\left(n, k\right) = 0 \Leftrightarrow r\cdot k \le n\).
    \item Opazujemo element \(n\). Lahko ga vzamemo v blok z \(r\) elementi, ali pa naredimo bloke iz
        ostalih in ga dodamo v enega izmed njih. To nam da rekurzivno zvezo
        \[S_r\left(n, k\right) = \binom{n - 1}{r - 1} S_r\left(n - r, k - 1\right) + k S_r\left(n - 1, k\right).\]
        Pri tem se zavedamo, kdaj je \(S_r\left(n, k\right) = 0\).
    \item \[S_3\left(10, 3\right) = \binom{9}{2} S_3\left(7, 2\right) + 3 S_3\left(9, 3\right) = \ldots = 2100 \]
    \begin{table}[!h]
        \centering
        \begin{tabular}{|p{0.1\textwidth}|p{0.1\textwidth}|p{0.1\textwidth}|p{0.1\textwidth}|}
            \hline 
            $\displaystyle n$\textbackslash$\displaystyle k$ & $\displaystyle 1$ & $\displaystyle 2$ & $\displaystyle 3$ \\
            \hline 
            $\displaystyle 1$ & $\displaystyle 0$ & $\displaystyle 0$ & $\displaystyle 0$ \\
            \hline 
            $\displaystyle 2$ & $\displaystyle 0$ & $\displaystyle 0$ & $\displaystyle 0$ \\
            \hline 
            $\displaystyle 3$ & $\displaystyle 1$ & $\displaystyle 0$ & $\displaystyle 0$ \\
            \hline 
            $\displaystyle 4$ & $\displaystyle 1$ & $\displaystyle 0$ & $\displaystyle 0$ \\
            \hline 
            $\displaystyle 5$ & $\displaystyle 1$ & $\displaystyle 0$ & $\displaystyle 0$ \\
            \hline 
            $\displaystyle 6$ & $\displaystyle 1$ & $\displaystyle 10$ & $\displaystyle 0$ \\
            \hline 
            $\displaystyle 7$ & $\displaystyle 1$ & $\displaystyle 35$ & $\displaystyle 0$ \\
            \hline 
            $\displaystyle 8$ & $\displaystyle 1$ & $\displaystyle 91$ & $\displaystyle 0$ \\
            \hline 
            $\displaystyle 9$ & $\displaystyle 1$ & $\displaystyle 210$ & $\displaystyle 280$ \\
            \hline 
            $\displaystyle 10$ & $\displaystyle 1$ & $\displaystyle 456$ & $\displaystyle 2100$ \\
            \hline
        \end{tabular}
    \end{table}
\end{enumerate}

\end{document}