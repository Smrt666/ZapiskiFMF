\documentclass[11pt]{article}

\usepackage{amsmath, amsthm, amssymb, amsfonts}
\usepackage[margin=3cm]{geometry}
\usepackage[utf8]{inputenc}
\usepackage[T1]{fontenc}
\usepackage[slovene]{babel}
\usepackage{thmtools}
\usepackage{lmodern}
\usepackage[colorlinks=false]{hyperref}
\usepackage{textcomp}
\usepackage{enumitem}


\title{Rešitve 1.\ domače naloge iz diskretne matematike}
\author{}
\date{}

\begin{document}

\maketitle

\section*{1.~naloga}
    Koliko je štirimestnih števil, pri katerih je vsota števk sodo število?
\subsection*{Rešitev}
    Enostavno lahko pokažemo, da je takih števil polovica. Definiramo izomorfizem, kjer vsakemu štirimestnemu številu 
    priredimo drugo štirimestno število, tako da spremenimo zadnjo števko. Zadnjo števko povečamo za \(1\) in 
    vzamemo ostanek po modulu \(10\). Tako dobimo bijekcijo med števili, ki imajo sodo vsoto števk in tistimi, ki
    imajo liho vsoto števk. Vseh štirimestnih števil je \(9000\), torej je število štirimestnih števil, ki imajo
    sodo vsoto števk enako \(4500\).

    Lahko bi se tega lotili tudi tako, da bi ločili števila glede na parnost prve števke in iskali načine, da ostale \(3\) števke 
    dajo sodo ali pa liho vsoto.

\section*{2.~naloga}
Koliko je različnih besed dolžine \(3\), sestavljenih iz črk \(A\), \(B\), \(C\), \(D\), \(E\) in \(F\), če \begin{enumerate}[label = (\alph*)]
    \item ponavljanje ni dovoljeno in besede vsebujejo vsaj eno od črk \(E\) in \(F\)?
    \item ponavljanje je dovoljeno in besede vsebujejo vsaj eno od črk \(E\) in \(F\)?
    \item ponavljanje ni dovoljeno in besede vsebujejo črko \(E\) in črko \(F\)?
    \item ponavljanje je dovoljeno in besede vsebujejo črko \(E\) in črko \(F\)?
\end{enumerate}
\subsection*{Rešitev}
\begin{enumerate}[label = (\alph*)]
    \item Vseh besed brez ponovljenih črk je \(\binom{6}{3} \cdot 3! = 120\). Nobena od črk \(E\) in \(F\) se ne pojavi v \(\binom{4}{3} \cdot 3! = 24\) besedah. 
        Skupno število besed z vsaj eno od teh dveh črk je torej \(120 - 24 = 96\).
    \item Vseh besed z ponovljenimi črkami je \(6^3 = 216\). Besed brez črk \(E\) in \(F\) je \(4^3 = 64\).
        Skupno število besed z vsaj eno od teh dveh črk je torej \(216 - 64 = 152\).
    \item V tem primeru imamo \(2\) možnosti za izbiro, v kakšnem vrstnem redu se pojavita \(E\) in \(F\) ter \(3\) možnosti, kje je tretja črka. Za tretjo črko imamo \(4\)
        možnosti. Skupaj je to \(2 \cdot 3 \cdot 4 = 24\) možnosti.
    \item Če se pojavita \(E\) in \(F\), lahko ločimo dve možnosti.
        Lahko se obe pojavita enkrat, lahko pa ena dvakrat. Če se obe enkrat, imamo \(2\) možnosti za njun vrstni red, \(4\) možnosti za izbiro tretje črke in \(3\) možnosti za 
        njen položaj v besedi. S tem dobimo \(2 \cdot 4 \cdot 3 = 24\) možnosti. Če se ena črka pojavi dvakrat, imamo \(2\) možnosti za izbiro katera črke se pojavi dvakrat
        in \(3\) možnosti za vrstni red teh črk. Zadnja možnost nam da \(2 \cdot 3 = 6\) načinov, skupaj za prvi primer je to \(24 + 6 = 30\) besed.
\end{enumerate}

\section*{3.~naloga}
    Naj bo
    \[A = \left\{\left(i, j\right); i, j \in \mathbb{N}, 20 \le i, j \le 40\right\} \text{in \ } B = \left\{(i, j) \in A; i + j \text{je sodo število}\right\}.\]
    Koliko elementov ima množica \(B\)?
\subsection*{Rešitev}
    Števili \(i\) in \(j\) morata biti ali obe sodi ali pa obe lihi. Lihih izbir imamo \(10\) in sodih \(11\). Skupaj je to \(10 \cdot 10 + 11 \cdot 11 = 221\).

\section*{4.~naloga}
Naj bo A množica z \(n\) elementi. \begin{enumerate}[label = (\alph*)]
    \item Koliko je vseh binarnih relacij na množici \(A\)?
    \item Koliko je vseh refleksivnih relacij na množici \(A\)?
    \item Koliko je vseh simetričnih relacij na množici \(A\)?
    \item Koliko je vseh refleksivnih in hkrati simetričnih relacij na množici \(A\)?
\end{enumerate}
\subsection*{Rešitev}
    \begin{enumerate}[label = (\alph*)]
        \item Binarna relacija je podmnožica \(A \times A\), torej ima \(2^{n^2}\) možnosti.
        \item Refleksivna relacija je podmnožica \(A \times A\), ki vsebuje diagonalno relacijo. Ta ima \(n\) elementov, torej je možnosti za izbiro preostalih elementov \(2^{n^2 - n}\).
        \item Simetrične relacije so v bijekciji z zgornje trikotnimi \(n \times n\) matrikami nad poljem \(\{0, 1\}\). (Vrednosti nad diagonalo določijo vrednosti pod diagonalo.) 
            Torej je možnosti \(2^{\frac{n(n + 1)}{2}}\).
        \item Glede na prejšnji primer izgubimo prostost diagonale, torej je možnosti \(2^{\frac{n(n - 1)}{2}}\).
    \end{enumerate}

\section*{5.~naloga}
    Ob železniški progi je \(k\) postaj. Koliko različnih vozovnic je treba pripraviti, da jih bodo
    imeli na razpolago za vse relacije (v obe smeri)? Kaj pa, če se mora vsak potnik peljati vsaj
    dve postaji?
\subsection*{Rešitev}
    Vsak urejen par različnih postaj je veljavna relacijo v neko smer in vsaka relacija je enolično določena z urejenim parom postaj. Torej potrebujemo
    \(k\left(k - 1\right)\) vozovnic. 
    
    Lahko gledamo, na koliko postajah lahko potnik izstopi, če začne na neki postaji. Potnik, ki je vstopil na 1.~postaji, lahko izstopi na \(k - 2\) postajah, na 2.~postaji na \(k - 3\) postajah, \ldots,
    na \(\left(k - 2\right)\)-ti postaji pa na \(1\) postaji. Za obe smeri skupaj je to 
    \[2 \cdot \left(\left(k - 2\right) + \left(k - 3\right) + \ldots + 1\right) = 2 \cdot \frac{\left(k - 1\right)\left(k - 2\right)}{2} = \left(k - 1\right)\left(k - 2\right)\] 
    vozovnic.
\end{document}