\documentclass[11pt]{article}

\usepackage{amsmath, amsthm, amssymb, amsfonts}
\usepackage[margin=3cm]{geometry}
\usepackage[utf8]{inputenc}
\usepackage[T1]{fontenc}
\usepackage[slovene]{babel}
\usepackage{thmtools}
\usepackage{lmodern}
\usepackage[colorlinks=false]{hyperref}
\usepackage{textcomp}
\usepackage{enumitem}


\title{Rešitve 2.\ domače naloge iz diskretne matematike}
\author{}
\date{}

\begin{document}

\maketitle

\section*{1.~naloga}
Na koncertu bo nastopilo \(n \ge 2\), pevcev in \(m\) pevk, vsak bo odpel po eno točko sporeda.
Na koliko načinov lahko sestavimo spored, če mora koncert začeti in končati pevec?
\subsection*{Rešitev}
    Najprej izberemo prvega in zadnjega pevca, kar lahko storimo na \(n \cdot (n - 1)\) načinov. 
    Vmes razporedimo preostale pevce in pevke, kar lahko storimo na \((n - 2 + m)!\) načinov.
    Skupno število možnosti je torej \(n \cdot (n - 1) \cdot (n + m - 2)!\).

\section*{2.~naloga}
Pajek ima za vsako od svojih osmih nog svojo nogavico in čevelj (vsaka nogavica in vsak
čevelj paše na točno eno nogo). Koliko je različnih vrstnih redov, na katere se pajek lahko
obuje, če mora na vsako nogo najprej obuti nogavico in potem čevelj?
\subsection*{Rešitev}
    Recimo, da imamo \(n\) načinov, kako se lahko pajek obuje. Če bi na kateri nogi zamenjal
    vrstni red nogavice in čevlja, bi bil ta neveljaven. Torej imamo za vsak dober vrstni red obuvanja
    \(2^8 - 1\) neveljavnih. Hitro lahko vidimo, da je vsak vrstni red dober, ali pa ravnokar opisane oblike.
    Skupno število veljavnih in neveljavnih vrstnih redov je torej \(2^8 \cdot n\), 
    kar je enako številu vseh vrstnih redov \(\left(2 \cdot 8\right)!\). Ti števili sta enaki, torej je \(n = \frac{\left(2 \cdot 8\right)!}{2^8}\).

\section*{3.~naloga}
Na koliko načinov lahko postavimo k trdnjav na šahovnico dimenzije \(m \times n\) tako, da se
paroma ne napadajo?
\subsection*{Rešitev}
    Enostaven primer je, če je \(n = m = k\). V tem primeru vidimo, da je v vsaki vrstici ena trdnjava.
    V prvi vrstici imamo \(k\) možnosti. S postavitvijo prve trdnjave izgubimo en stolpec, eno vrstico in eno trdnjavo.
    Imamo enake pogoje kot prej, torej imamo \(k - 1\) možnosti za postavitev druge trdnjave. Očitno je, da je število možnosti
    za ta primer enako \(k!\).

    V splošnem bomo morali izbrati \(k\) vrstic in \(k\) stolpcev, kamor bomo postavili trdnjave.
    To lahko storimo na \(\binom{m}{k} \cdot \binom{n}{k}\) načinov. Ko so ti izbrani, dobimo prvi primer, ki smo ga rešili.
    Skupno število možnosti je torej 
    \[\sum_{k = 0}^{\min\{m, n\}} \binom{m}{k} \cdot \binom{n}{k} \cdot k!.\]

\section*{4.~naloga}
Deset turistov in deset turistk si želi ogledati Blejski otok. Na koliko načinov si ga lahko
ogledajo s \(5\) enakimi čolni, če morata v vsakem čolnu sedeti po dva moška in po dve ženski?

\subsection*{Rešitev}
    Iščemo število parov, ki jih lahko dobimo iz parov turistov in turistk.
    Če imamo poparčkane turiste, lahko damo te pare v vrsto na \(5!\) načinov in jih zamenjamo znotraj parov na \(2^5\) načinov. 
    Tako dobimo vse možne permutacije turistov, ki jih je \(10!\). Sledi, da je parov turistov in turistk \(\frac{10!}{5!2^5}\). 

    Ko imamo enkrat pare turistov in turistk, naredimo pare iz njih. Petim parom turistov pripišemo
    po en par turistk, kar lahko storimo na \(5!\) načinov. Skupno število možnosti je torej 
    \[{\left(\frac{10!}{5! \cdot 2^5}\right)}^2 \cdot 5!.\]

\section*{5.~naloga}
K brivcu vkoraka nogometno moštvo (\(11\) mož), v čakalnici pa je ravno \(11\) stolov. Na koliko
načinov se lahko posedejo, če Andrej, Bojan in Cene ne želijo sedeti skupaj? Kaj pa, če želi
Bojan sedeti med Andrejem in Cenetom (ne sedijo nujno skupaj)?
\subsection*{Rešitev}
    V prvem primeru bodo glede na drug drugega sedeli Andrej, Bojan in Cene na enega od \(3!\) načinov.
    Na dveh mestih med njimi je po še vsaj en stol. Preostalih \(6\) mest lahko vrinemo kamorkoli na začetek, konec, 
    ali pa med dva izmed njih. To vrivanje lahko storimo na število šibkih kompozicij števila \(6\) na \(4\) dele,
    kar je \(\binom{6 + 4 - 1}{4 - 1} = \binom{9}{3}\). Na vsa ta mesta razporedimo še ostalih \(8\) ljudi. Končen odgovor je torej
    \[3! \cdot \binom{9}{3} \cdot 8!.\]

    V drugem primeru bomo najprej postavili Bojana med Andreja in Ceneta. To lahko storimo na \(2\) načina.
    Zanje moramo izbrati \(3\) sedeže, kar lahko storimo na \(\binom{9}{3}\) načinov. Preostalih \(8\) ljudi razporedimo na preostalih \(8\) mest.
    Skupno število možnosti je torej
    \[2! \cdot \binom{9}{3} \cdot 8!.\]

\section*{6.~naloga}
Na piknik je prišlo \(5\) matematikov, \(7\) fizikov in \(3\) kemiki. Na koliko načinov se lahko postavijo
v vrsto za pleskavice, če so vsi matematiki pred vsemi fiziki?

\subsection*{Rešitev}
    Matematike lahko postavimo v vrsto na \(5!\) načinov, fizike na \(7!\) načinov in kemike na \(3!\) načinov. 
    Matematiki so pred fiziki, na poljubna mesta pa vrinemo \(3\) kemike.
    To lahko storimo na \(\binom{3 + \left(5 + 7 + 1\right) - 1}{\left(5 + 7 + 1\right) - 1} = \binom{15}{12}\) načinov. Skupno število možnosti je torej
    \[5! \cdot 7! \cdot 3! \cdot \binom{15}{12}.\]

\section*{7.~naloga}
Na koliko načinov lahko postavimo v vrsto \(5\) rdečih, \(7\) modrih in \(3\) zelene kroglice, če mora
biti vsaka rdeča kroglica pred vsemi modrimi kroglicami?
Pripomba: med kroglicami iste barve ne ločimo.

\subsection*{Rešitev}
    Naloga je podobna prejšnji, le da nas ne zanimajo razporeditve znotraj skupin.
    Odgovor je kar \(\binom{15}{12}\).

\section*{8.~naloga}
Profesor je predaval \(n\) let. Vsako leto je povedal na predavanjih \(k\) anekdot. Vsaj koliko
anekdot je moral poznati, če v dveh različnih letih ni povedal istih \(k\) anekdot? Rešite nalogo
še za primer \(n = 10\) in \(k = 4\).

\subsection*{Rešitev}
    Naj bo \(A\) množica vseh anekdot, ki jih je profesor povedal. Naj bo \(B\) množica vseh možnih \(k\)-tih podmnožic množice \(A\).
    Poiskati moramo najmanjši \(|A|\), da bo \(|B| \ge n\). To je ekvivalentno iskanju najmanjšega \(|A| = x\), da bo \(\binom{x}{k} \ge n\).
    
    Za \(n = 10\) in \(k = 4\) dobimo, da je \(x = 6\), saj je \(\binom{6}{4} = 15 \ge 10\) in \(\binom{5}{4} = 5 < 10\).
\end{document}